% Created 2020-04-21 Tue 15:39
% Intended LaTeX compiler: pdflatex
\documentclass[11pt]{article}
\usepackage[utf8]{inputenc}
\usepackage[T1]{fontenc}
\usepackage{graphicx}
\usepackage{grffile}
\usepackage{longtable}
\usepackage{wrapfig}
\usepackage{rotating}
\usepackage[normalem]{ulem}
\usepackage{amsmath}
\usepackage{textcomp}
\usepackage{amssymb}
\usepackage{capt-of}
\usepackage{hyperref}
\author{Filipa  Calado}
\date{\today}
\title{}
\hypersetup{
 pdfauthor={Filipa  Calado},
 pdftitle={},
 pdfkeywords={},
 pdfsubject={},
 pdfcreator={Emacs 26.2 (Org mode 9.1.9)}, 
 pdflang={English}}
\begin{document}

\tableofcontents

This dissertation looks at new ways of reading queer bodies and
experience within technological contexts. How do digital tools and
platforms change the way we interact with affect (associated with
queer experience) within texts? What about the digital allows us to
create sensational reading experiences? In other words, how does media
touch us, and how might we touch what we read?

This chapter proposes a reading methodology that leverages the
critic's relation to the text to open possibilities for interpretation
and connections to the textual material. It explores the ways that
reading practices across two different fields intertwine, and how this
creates a new method for reading queer narratives in digital
contexts. 

\subsection{Touch: a new way of reading: Sedgwick}
\label{sec:orge277c56}

If digital humanists and queer theorists are going to find some common
ground, they might as well start with \emph{touch}. Touch is driven by a
desire for connection, for correlation, for linking one "body" to
another. The root of the word digital, "digitus," comes from the Latin
word for finger or toe. Our digits extend forth, the furthest
appendage still considered part of the coherent body, as the first
line of contact with the world. Touch goes both ways; what I touch
also touches me; one body impressed by or in collision with
another. The sensation of touch often obscures this dual effect, and
some bodies appear to desire touching rather than being
touched. Queerness is similarly concerned with contact and desire, and
more precisely with the desire for contact. Queer contact connotes not
traditional contact, but contact that opposes or even offends
expectations. It is to the side of the acceptable, bounding the
confines of the normative.\footnote{What is "queer"? How do we define "queer"? Queer is the feeling
I get when I'm reading something that I'm not supposed to.} Both queer and digital are out there,
our frames for engaging with the outside world. Both queer and digital
are alternative ways of engaging with the environment.

Touch either keeps or recuperates the inherent connection between
bodies. All things touch each other, something that heteronormativity
tries to suppress. For things to not touch, to be severed or
"objectified," moves them into a relation of violence. Gloria Anzaldua
explains that separation is brutal: "In trying to become 'objective,'
Western culture made 'objects' of things and people when it distanced
itself from them, thereby losing 'touch' with them. This dichotomy is
the root of all violence" (37). Losing touch is a prerequisite for
exploitation. The sundering of "objects" from our touch primes us to
take advantage of them. Colonial history is a case study in losing
touch: "White America has only attended to the body of the earth in
order to exploit it, never to succor it or to be nurtured in it"
(68). Anzaldua's \emph{mestiza}, birthed in the open wound of the border,
"where the Third World grates against the first and bleeds," is an
attempt to bring together what has been separated (3). Those who live
on the border know better than anyone--divisions between bodies puts
those bodies into conflict.

Touch offers myriad ways of relation. Eve Kosofsky Sedgwick offers
touch as a way of connecting to objects that evades "dualistic
thought," that is, in "binary" thought, where things are presumed to
be discrete and opposed. Sedgwick explains that "the sense of touch
makes nonsense out of any dualistic understanding of agency and
passivity; to touch is always already to reach out, to fondle, to
heft, to tap, or to enfold" (13). Touch is notable for what it manages
to avoid---the upper hand, the either/or. Touch opens a range of
relations where power is not always reduced to opposition. Touch opens
relations to the outside world. There is more than one kind of touch.

As an antidote, touch is not a naturally western mode of
relation. Touch is queer. Touch can inflect the way we read and talk
about what we read.


\subsection{The problem: Queerness as untouchable}
\label{sec:org9bb7448}
In some queer theory and digital humanities runs a similar hesistation
not to overdetermine or overinterpret the content of what we read. For
queer theory, this hesitation pushes against the trend of locating
pain in order to redeem it or reveal its workings. This reading
practice (which has been called "suspicious reading" or "paranoid
reading"\footnote{Rita Felski? and Eve Kosofsky Sedgwick.}) seeks to expose the effects of homophobic prohibition
and repression, often with the goal of affirming queer subjects or
recuperating their losses. Reading is oriented around finding and
exposing these aspects about queer experience (the pain and shame of
the closet, for example) and turning those aspects into sites of
political resistance, liberation, or pride. [NEEDS A QUOTE, SEDGWICK,
THAT EXPLAINS PARANOID READING]

\subsubsection{Heather Love's \emph{Feeling Backward}}
\label{sec:org9a0116e}
In resisting the temptations to redeem or console psychic suffering by
queer subjects, Heather Love offers an alternative reading strategy,
"feeling-backward." This strategy opens a space for bad feelings
without trying to recuscitate, justify, or transform them. She focuses
on feelings such as "nostalgia, regret, shame, despair,
\emph{ressentiment}, passivity escapism, self-hatred, withdrawal,
bitterness, defeatism, and loneliness," which, according to Love, are
tied to "the historical impossibility of same-sex desire" (4, emphasis
original). She examines the burdened protagonists from famous
modernist texts like Walter Pater's \emph{The Renaissance: Studies in Art
and Poetry} (1873), Radclyffe Hall's \emph{The Well of Loneliness} (1928),
Willa Cather's \emph{My Ántonia} (1918), and Sylvia Townsend Warner's
\emph{Summer Will Show} (1936). Love argues that the shame and stigma
experienced by these characters ought to be recognized rather than
resolved. Instead of turning negative histories into sites of
resistance or affirmation, these hurting characters might have full
reign over their own darkness. And this darkness must be where the
critic will meet them.

\subsubsection{identification is risky}
\label{sec:orgf7b4887}
As readers, identifying with literary subjects is both dangerous and
seducticve. Identities within texts are not stable across time and
place, and acts of identification might collapse or overlook the
complexity of experience. For queer readers in particular,
identification often emerges from a desire to recognize within the
past something that affirms queer experience in the present. Love
describes queer critics, "Like demanding lovers [who] promise to
rescue the past when in fact they dream of being rescued themselves"
(33). Reading in this sense is a search for reflection, community, or
similitude, a link between past and present. When identification is
possible, however, it can be shattering. In "The Torment of Queer
Literature," Kelly Caldwell explains the quandry of reading James
Baldwin's \emph{Giovanni's Room} as a transgender woman: "what if the only
available act of identification is one of stigma and shame? Embracing
queerness is often embracing abjection. Sometimes identification is
loss and despair" (par. 4). Identification tends to center around
these "bad feelings" which offer less fodder for political
resistence. However, identification with more positive aspects of
queer experience is hardly an alternative. For many readers, the more
redemptive or celebratory narratives offer no consolation. The reader
is stuck between recognizing their own pain or feeling guilty for not
recognizing pleasure: "Either read a book like \emph{Giovanni’s Room} at
the risk of recognizing David’s denial and repression as my own, or
read a book that celebrates queer lives and sex boldly and end up
despising my own cowardice" (par. 17).

\subsubsection{queerness is a failed project, which is why it's so tempting.}
\label{sec:org1d3c91a}
The more hopeless and resistant queer subjects make for more tempting
identifications. Love explains how these subjects remain beyond the
reader's grasp: "As queer readers we tend to see ourselves as reaching
back toward isolated figures in the queer past in order to rescue or
save them. It is hard to know what to do with texts that resist our
advances" (8). The reason that these subjects remain so unreachable
has to do with the nature of queerness itself, which represents
absence, loss, and failure. Love illustrates this quality by evoking a
Greek myth, Orpheus and Eurydice, in which the lover botches his
beloved's rescue by looking back at her as they exit the
underworld. Love quotes from Maurice Blanchot's account of the story
in "The Gaze of Orpheus," to describe what Orpheus searches for in the
prohibited and doomed glance backward:

\begin{quote}
Not to look would be infidelity to the measureless, imprudent force
of his movement, which does not want Eurydice in her daytime truth and
in her everyday appeal, but wants her in her nocturnal obscurity, in
her distance, with her closed body and sealed face---wants to see her
not when she is visible, but when she is invisible, and not as the
intimacy of familiar life, but as the foreignness of what excludes all
intimacy, and wants, not to make her live, but to have living in her
the plenditude of death. 50
\end{quote}

Orpheus's downfall is his desire for a glimpse at what cannot be
grapsed, at what remains beyond the light. This desire is not for
"daytime truth" but for "noctural obscurity," which is always receding
at the moment of pursuit. Like Eurydice, queerness emerges only to
slip away, turning its face from the parched gaze. Can we be blamed
for looking for that which cannot be grasped? No, because queerness
has always been structured by that which is not, by what Love calls
"impossible love" (24). Not only is queerness projected to fail, it is
a project of failure. Love reminds us that "Queer history has been an
education in absence" (50). In learning failure and loss, queer
readers can only identify with what they have been taught to recognize
as untouchable. Full identification, like Eurydice in the daylight, is
prevented by design.

\subsubsection{identifying, but not fully.}
\label{sec:org882faa6}
Love proposes a method in which the goal is not to redeem queer
subjects or resolve queer failure. Rather, the problem of
identification is turned to a reading strategy: "I want to suggest a
mode of historiography that recognizes the inevitability of a 'play of
recogniztions,' but that also sees these recognitions not as consoling
but as shattering" (45). Reading, for Love, can enact a "play of
recognitions," which is a way of making fleeting connections that do
not presume complete understanding. It is a way of identifying, but
not fully. Full identification would attempt to wrench the subject
from its suffering, and effectively transform it into something
else. Rather that attempt to rescusitate it, Love looks to the ways
that identity unsettles and dissolves subjectivity. She gives the
example of Stephen Gordon from Radclyffe Hall's \emph{The Well of
Loneliness}. Once considered too depressing as a model of lesbianism,
recent critics have cast Stephen Gordon as a transgender figure. Love
resists this label, maintaining that Stephen is “beyond the reach of
such redemptive narratives” (119). The question, for Love, is not
whether Stephen is a pre-op FTM (Female-to-Male), but how Stephen’s
existential negativity can be read as an embodied phenonmenon, as “a
social experience insistently internalized and corporeal” (108).

\subsubsection{Critique of affirmation: Cvetkovitch}
\label{sec:org35847bc}

Attempts to affirm negative queer experience can be harmful. Ann
Cvetkovitch's work on trauma studies provides an example of how this
tendency can create further misunderstanding about suffering. In her
book, \emph{Archive of Feelings}, Cvetkovitch explores expressions of
trauma within the public sphere. She asks how individuals might
reclaim some of the most negative and traumatic feelings into
something positive and theraputic: "I want to place moments of extreme
trauma alongside moments of everyday emotional distress that are often
the only sign that trauma's effects are still being felt” (3). She
wrests trauma studies out of medical discourse and into public
culture---turning something that is traditionally private and
pathologized into something communitarian, an open, everyday "archive
of feelings."

Importantly, Cvetkovitch marshalls this reconfiguration of trauma to
expand what we consider the 'archive'. She also makes some incisive
points about the inability to fully portray suffering: "Because trauma
can be unspeakable and unrepresentable and because it is marked by
forgetting and dissociation, it often seems to leave behind no records
at all" (7). Cvetkovitch explores alternative methods of figuring
trauma, which are transformed when they enter the public sphere. She
cites examples from public performances like rock shows or
documentaries, in which the artists enact "moments of intense affect
that are transformative or revealing” (26). 

Although her focus on the affective and ephemeral dimension of
performance opens up conceptions of the archive, Cvetkovitch perhaps
goes too far when she suggests that these performances are redemptive
or in some way compensate for traumatic experience. She indicates that
such performances go so far as to alleviate psychological damage and
suffering: “Imaginative work that may bear an oblique relation to the
actual event of sexual abuse can ultimately be more ‘healing’ than an
explicit rendering of the event” (94). Trauma is a real medical
condition, with real consequences (death) for those who do not seek
treatment or downplay its life-threatening effects. Critics should be
careful in extending a definition of trauma that will end up hurting
those who are affected by it. We do not need to move trauma strictly
from the medical discourse in order to have a more communitarian,
open, and public relationship to it. There are other ways to confront
stigma which doesn’t attempt to redeem it, as Love explores with her
notion of “Feeling Backward.”

The step that Cvetkovitch takes with regard to trauma is interesting,
however, for what it suggests about the role of the critic in
analysis. The point isn't to find evidence of overcoming queer
suffering, but to examine the ways that queerness is figured in
abstraction. What does queerness look like, what can it do? 


\subsection{The problem: Data as cooked}
\label{sec:org01f241f}
As a mode of relationality, "Feeling Backward" not presume a full
connection between the critic and subject, keeping the subject at arms
length. It approaches queerness as something receding, even when the
critic is perpetually in pursuit. This relationship between critic and
textual subject evokes some of the attitudes that digital humaninists
take toward their data. Critics such as Johanna Drucker and Ted
Underwood are careful to qualify the nature of data as constructed,
wrenched from the reality of lived experience, and necessarily reduced
to fit whatever environs required by analysis.

\subsubsection{Drucker's skewing the graphs}
\label{sec:orgd87e1d0}

Johanna Drucker argues that quantification techniques (such as
visualizations in graphs and charts) actually misrepresent the data
they are meant to convey. Drucker explains that, in order to place
this data on a graph or chart, it undergoes a
transformation. Complexity is reduced to whatever quality the
visualization apparently requires. To illustrate this reduction,
Drucker presents a chart displaying the amount of books published over
several years. The chart appears to convey production during this
specific time period\footnote{Drucker implicitly refers to the first chapter from Franco
Moretti's \emph{Graphs, Maps, Trees} (2007), throughout which Moretti
graphs novels by their publication date between 1700 and 2000 and
draws conclusions about the relationship between genre and generations
of readers.}, but Drucker explains that publication date
is an arbitrary metric for capturing production. She brings to the
surface all the assumptions made in such a metric, for example, the
limitations of "novel" as a genre and the connotations behind
"published," which suggests date of appearance, but has no indication
of composition, editing, review, distribution. Drucker reminds us that
each piece of data carries with it the result of many interpretive
decisions, which carry with them varying degrees of opacity. These
interpretations ("reductions") are necessary in order to present
complex concepts like book production as a bar on a chart. Drucker
explains: "the graphical presentation of supposedly self-evident
information (again, formulated in this example as “the number of
novels published in a year”) conceals these complexities, and the
interpretative factors that bring the numerics into being, under a
guise of graphical legibility" (Drucker par. 23).

To resist the reductions of "data," a term that connotes that which is
"given," Drucker proposes "capta," to suggest the act of being taken
and transformed. Drucker's "capta" is deliberately creative, turning
graphical expressions into expressive metrics: components used for
measurement, like lines or bars on a graph, break or are fuzzy and
permeable. Objects are not discrete entities, but interact with the
other objects in the visualization. For example, in a bar graph of
book publications/year, she warps the bars on the graph, making some
of them fuzzy, wider, shorter, in an attempt to show that publication
as a metric elides other information such as composition, editing,
purchasing, etc.

This activity is a way of figuring elements that have been reduced,
resolved, or ignored in traditional quantitative analysis. It evokes
what Love says about queer subjectivity and experience being beyond
the reaches of the critic. Drucker makes evident what is overlooked or
assumed when dealing with complex subjects. She places those elements
there, for all to see, in a way that muddles (rather than
simplifies\footnote{Moretti: "'Distant reading'\ldots{} where distance is however not an
obstacle but /a specific form of knowledge" (1).}) the relationship between them. She does try to
figure these elements, but not in a way that attempts to clarify or
resolve their complexity. Rather, like Love, she works on the “image
of exile, of refusal, even of failure” (Love 71).

\subsubsection{Ted Underwood's models as object of study}
\label{sec:org3e97dd0}

Ted Underwood and other literary critics doing Computational Literary
Studies (CLS) approach their data with vastly different
commitments. Underwood harnesses computational power and
sophistication to glimpse the big picture of literary history, what he
calls the "distant horizon" of literary trends across centuries. His
argument convincingly begins with the observation that human
capacities---sight, attention, and memory---preclude them from
grasping the larger patterns of literary history over time
periods. Distant reading, whereby "distance" implies abstraction, or
the simplification of textual data into computable objects such as
publication dates and genres, allows critics to make connections in
apparent chaos, to draw a steady line of historical development
through the swarm of overflowing information. According to Underwood,
distant reading opens new scopes to literary analysis, which would
otherwise be invisible to readers: "a single pair of eyes at ground
level can't grasp the curve of the horizon" (x).

Though to a much lesser degree than Drucker, Underwood similarly turns
his computational method into an object of study. His research deploys
machine learning, that is, computer programs "trained" by certain data
sets to make predictions about other datasets. Underwood studies how
"models," or calculations based on multiple variables, created by
sample data can then be used to measure further data. One of his
models measures the probability that computers can guess the sex of
a fictional character based on the words associated with that
character. Underwood explains how the test is configured:

\begin{quote}
We represent each character by the adjectives that modify them, verbs
they govern and so on---excluding only words that explicitly name a
gendered role like \emph{boyhood} or \emph{wife}. Then, we present characters,
labeled with grammatical gender, to a learning algorithm. The
algorithm will learn what it means to be 'masculine' or 'feminine'
purely by observing what men and women actually do in stories. The model produced by the algorithm can make predictions about other
characters, previously unseen. 115
\end{quote}

The computer takes in information about some (the more the better)
books and studies that information in order to make predictions about
other books. The resulting model, therefore, is always guided by its
previous experience. Underwood rightly points out that such
calculations cannot be taken as fact. Like humans, "machine learning
tends to absorb assumptions latent in the evidence it is trained on"
(xv). To Underwood, machine learning is no more "objective" than
regualar analysis. This is why Underwood calls his work "perspectival
modeling," where he studies how datasets reveal, not the truth of
literary histroy, but the \emph{approaches} of those who study it: "By
training models on evidence selected by different people, we can
crystallize different social perspectives and compare them rigorously
to each other" (xv).

The results of the analysis is baked into the process, something that
Underwood understands and accepts as part of the obstacles toward his
distant horizon. In looking at the way gender is characterized, or
rather how perspectival models characterize gender, in novels from the
18th century to the 21st, he finds that the results reproduce some of
the structuring assumptions from the outset. His examination of gender
characterization finds that "while gender roles were becoming more
flexible, the attention actually devoted to women was declining"
(114). The analysis points to a steady overapping of words used to
describe men and women over time, shown as a convergence on the graph
between words previously associated with women, such as "heart," which
begin to intersect with words typically assoicated with men, like
"passion," toward the middle of the 20th century. However, while the
categories of "masculine" and "feminine" words are progressively
blurred over time, the actual number of female \emph{characters}
declines. Underwood explains this drop could be due to several
reasons, one of which is the simple fact that the practice of writing
"gentrified" through the 20th century, when writing became
acknowledged and pursued as a male occupation (137). His analysis
shows that men tend to write more about men, while women write equally
about men and women. With less women writing, the amount of female
characters therefore declines. This explains how Underwood's seemingly
paradoxical conclusion, that gender roles become more flexible while
the actual prevalence of women dissapates from fiction, might be
possible. But Underwood also admits that another factor---the
assumption of gender as a binary category---might very well be guiding
his results: "One possible conclusion would be that the structural
positions of masculine and feminine identity, vis-'a-vis each other,
have remained very stable---while the actual content of masculinity
and femninity has been entirely mutable" (140). Viewing gender as a
binary construction perpetuates the structural categories of
male/female in a way that is at odds with the actual content of such
categorie. In other words, if gender is binary, then it stands to
reason that the relation between male and female will be one of
opposition. Underwood proposes that one way around this confining
structure of binary gender would be to refigure gender "as a spectrum
or as a \emph{multiplication} of gender identities that made the binary
opposition between masculine and feminine increasingly irrelevant to
characters' plural roles" (140).


\subsection{Critique of Reproducibility}
\label{sec:org023453a}
Scholars like Da, Underwood and Altschuler and Weimer who want something
reproducible, this overlooks the performativity of engaging with texts
online. 
\subsubsection{nan Z da on reproducibility}
\label{sec:orgd287e7e}
In a controversial peice about text analysis, Nan Z. Da deems
Computational Literary Studies (CLS) ineffective when her own
experiments fail to reproduce or verify the results of her
colleagues. Her emphasis on the “reproducible” in CLS extends one of
distant reading early champion's originating call for a “falsifiable
criticism”: both advocate for a methodology that is as reliable and
verifiable as the social sciences. I understand that Da uses
"reproducible" to mean analyses that can be copied and rerun by other
scholars in order to test the original result. This (though boring)
act is not the object of my critique. The object of my critique is the
insinuation in \emph{reproducible} that somehow these analyses exist by
themselves, outside of the critic.

\subsubsection{Ted Underwood on studying models}
\label{sec:orga452f3f}

Underwood from PMLA example. He only reproduces the gender
binary. Interweave with what Sedgwick says about binaries.

Ted Underwood puts forth a good understanding on quantification being
no more more objective than words. Rather than using "distant reading"
to ascertain "facts" about literary history, he's examining what
models are doing when used by humans. His focus on "perspectival
modeling" reveals how the computer process reproduces human
assumptions.

However, the results of the analysis are always baked into the start,
ask Underwood's conclusions about gender reveal gender to be a binary
and oppositional force.


\subsection{For alternative readings: queer theory}
\label{sec:org1d2a005}
\subsubsection{Felski's post critical reading: the illusion of emotional detachment}
\label{sec:org6bec89f}
The reality is that we are stuck in these bodies of our thinking. Rita
Felski describes how seemingly neutral and detatched critical stance
belies an emotional disposition:

\begin{quote}
Scholars like to think that their claims stand or fall on the merits
of their reasoning and the irresistible weight of their evidence, yet
they also adopt a low-key affective tone that can bolster or
drastically diminish their allure. Critical detachment, in this light,
is not an absence of mood but one manifestation of it---a certain
orientation toward one's subject, a way of making one's argument
matter. 6
\end{quote}

The "low-key affective tone" of scholarly discourse suggests that
affect, and the feeling subject associated with it, has been left out
of the critical process. However, appealing to the apparently
unemotional does not succeed in removing emotion from argument---this
is impossible---but it does reinforce the illusion that emotions don't
belong in rational thought. Actually they do---though the emotions of
critical discourse are of a quality and degree that mask their own
presence. Felski explains that, “Rather than an ascetic exercise in
demystification, suspicious reading turns out to be a style of thought
infused with a range of passions and pleasures, intense engagements
and eager commitments” (9). One follows the exposition of the framing
paradigms, the twists and turns of the driving question, the climax of
of discovery followed by the of denouement of the conclusion, one
immediately senses the full dramatic repertoire of critical
inquiry. 

And the illusion of reason as being devoid of emotion is not limited
to verbal discourse. It also pervades--perhaps even more
insidiously---the apparently objective reprsentations data
visualization. Graphs, charts, and maps all contain persuasive
elements that succeed through their invisibility, in the trust, for
example, that the souces are truthfully represented in the
visualization or the implied preference of some metrics over
others. Lauren Klein and Catherine D'Ignazio point out that "so-called
'neutral' visualizations that do not appear to have an editorial
hand\ldots{} might even be the most perniciously persuasive visualizations
of all!" (\emph{Data Feminism}, chapter 2). Not dots on a graph can be said
to be removed from the predelictions of the creator and the generosity
of the viewer.

\subsubsection{Felski \& Sedgwick affective approaches}
\label{sec:org979d359}

Critics like Rita Felski and Eve Sedgwick adopt an alternative
approach toward reading that exposes knowledge as derived from
embodied experience. Felski talks about reading as an affective
orientation, where readers position themselves and their desires
around texts. Felski critiques the popular orientation in literary
criticism centered on what Paul Riceour has called the “hermeneutics
of suspicion”---the desire to unmask and demystify the secrets of
literary works. According to Felski, critics generally behave as if
language is always withholding some truth, that the critic’s task is
to reveal the unsaid or repressed. She identifies the affective modes
of suspicion to include disenchantment, vigilance, paranoia. 

Sedgwick makes a similar assertion about tendencies of "paranoid
reading," though she bases her critique on Michele Foucault's
repressive hypothesis from his \emph{History of Sexuality, Vol. 1}, which
approaches discussions on sex and sexuality through the lense of
repression or prohibition. Rather than excavating the workings of the
repressive hypothesis, Foucault is interested in the ways that
discourse on sex has proliferated, in its multiplications that avoid
censure while satisfying the desire for sexual discourse. Left with no
place to go, discussion on sex simply continued to spread by
transforming itself into palatable discourses such as Marxism,
pyschoanalytic, libertarian, etc. By looking for the specter of
sex/power dynamics in these discourses, Foucault seems to work outside
the logic of the repressive hypothesis. But this is not the
case. Sedgwick explains that, "the almost delirious promise of the
book" is "the suggestion that there might be ways of thinking around
[the repressive hypothesis]" (9). In fact, Sedgwick explains that
Foucault's inquiry has been, from the start, structured by repression
and prohibition. She finds that the "critical analysis of repression
is itself inseparable from repression" (10). 

Felski and Sedgwick see a dead end in militant reading practices. 

Felski's nightmare: 
Sedgwick's wish: 

"How do we step outside the repressive hypothesis "to forms of thought
that would not be structured by the question of prohibition in the
first place?" (\emph{Touching Feeling} 11).

Speaking on Foucault's repressive hypothesis: "I knew what I wanted
from it: some ways of understanding human desire that might be quite
to the side of prohibition and repression, that might hence be
structured quite differently from the heroic, 'liberatory',
inescapably dualistic righteousness of hunting down and attacking
prohibition/repression in all its chameleonic guises" (\emph{Touching
Feeling} 10).

Felski shows how this suspicion toward texts forecloses other possible
readings while providing no guarantee of rigorous or radical
thought. Rather than adopt a suspicious attitude, Felski suggests that
literary scholars try “postcritical reading," which looks to what the
text suggests or makes possible. Felski wonders what if we allowed
ourselves to be marked or struck by what we read. Then, rather than
just be a cognitive activity, reading can become an “embodied mode of
attentiveness that involves us in acts of sensing, perceiving,
feeling, registering, and engaging” (176).

Reading is about movement 


Postcritical Reading --- "Reading, in this light, is a matter of
attaching, collating, negotiating, assembling—of forging links between
things that were previously unconnected”… “Reading, in this sense, is
not just a cognitive activity but an embodied mode of attentiveness
that involves us in acts of sensing, perceiving, feeling, registering,
and engaging” (176).

\subsubsection{Sedgwick on generative shame}
\label{sec:orge84f7f9}
What if we read Henry James mobilizing shame as a creative resource?
  For many queer people, shame is a structuring force in their
  identity. But this doesn’t mean we need to be negative, we can look
  to the ways that shame unlocks creativity and productivity---to the
  ways that metaphors are made possible through shame. James’
  “blushing”, “flushing” is linked to a fantasy of the skin being
  entered, or touched by a hand. GLOVE, GAGE, GAGEURE…  We can reclaim
  a negative affect of shame and approach it as a generative force.
\begin{itemize}
\item "Shame interests me politically, then, because it generates and
legitimates the place of identity--the question of identity--at the
origin of the impulse to the performative, but does so without
giving that identity space the standing of an essence. It
constitutes the as-to-be-constituted, which is also to say, as
already there for the (necessary, productive) misconstrual and
misrecognition. Shame--living, as it does, on and in the face--seems
to be uniquely contagious from one person to another. And the
contagiousness of shame is only facilitated by its anamorphic,
protean susceptibility to new expressive grammars" (63).
\end{itemize}







\subsection{Paralleling Queer \& DHers looking for alternative readings}
\label{sec:org53556d2}
\subsubsection{Case in point: klein's figuring the absence}
\label{sec:orge8a1fb4}
Draw Klein and Hartman together---this is what I want to do for Queer
texts. 


\subsection{Performativity}
\label{sec:org35cd4ce}
Digital formats and interfaces facilitate queer encounters methods, an
intimate process of engaging with literature on a computer, where
users can manipulate and transform text.
\subsubsection{Bode's materiality, critque of Underwood}
\label{sec:org5e16b4b}

Katherine Bode's critique of Underwood points out that QLS methods
incorporate hidden assumptions about the data, about what is
findable. She offers a method that builds off the humanistic
approaches in textual scholarship and bibliography, where the model is
prior to computation. 

\subsubsection{Tanya Clement: discovery}
\label{sec:org18d9e42}

\subsubsection{Against reproduction, for remediation/deformance}
\label{sec:orgd3cbd22}

\subsubsection{McGann's "prosthetic extension"}
\label{sec:orgf534b7b}
These tools work alongside the reader’s intuition, in what Jerome
McGann calls a “prosthetic extension of that demand for critical
reflection,” by which the reader is able to feel her way through the
text (18).

\subsubsection{Critique of Underwood's "sensitivity"}
\label{sec:orge8a21fd}

Underwood overlooks the ways that distant reading can be a
prosthesis. Claims that Quantitative are not as "sensitive" or
"exacting" as close reading, and are mostly useful for long views. How
can we approach distant reading as multiplying alternative readings?
Rightly points out that human attention guides the scale of
analysis. So we have to be very careful at the question we are posing,
and the way that we interact with the computer.
\begin{itemize}
\item "Critics who want to sensitively describe the merits of a single
work usually have no need for statistics\ldots{} Computational
analysis of a text is more flexible than it used to be, but it
is still quite crude compared to human reading; it helps mainly
with questions where evidence is simple too big to fit in a
single reader's memory" (xxi).
\end{itemize}
\begin{itemize}
\item Repeatedly stresses that the point of quantitative methods is to
discover new scales of analysis, but he seems to be looking for an
overarching theory that will encapsulate literary
history. Quantitative methods seek to overcome a problem of
attention, of memory, in order to gain a large view. Here, human
memory is a hindrance, rather than a drive. The goal is rather to
multiply alternative readings. 
\begin{itemize}
\item Attention determines analysis, analysis determines knowledge,
knowledge determines disciplines, periodization (8).
\item "The challenge is to find a perspective that makes the descriptions
preferred by eighteenth-, nineteenth-, and twentieth-century
scholars all congruent with each other" (32).
\end{itemize}
\end{itemize}


\subsection{Vantanges}
\label{sec:orgc1fd73d}

\subsubsection{Klein, Mandell, Caughie, Gaboury}
\label{sec:orga15ae4d}
\subsubsection{Against totalization}
\label{sec:orgaf7b275}
\subsubsection{The visible and the invisible, opting out}
\label{sec:org58e091b}


\subsection{Provisionality}
\label{sec:org417dd18}

\subsubsection{Susan Brown's provisionality}
\label{sec:orga7e1411}
\subsubsection{Julia Flander's work on Orlando}
\label{sec:org90f4113}
\subsubsection{Against stability}
\label{sec:org4e3ad55}


\subsection{Digital projects based on text manipulation:}
\label{sec:org12ac6e4}
I find that the haptic and exploratory activity of working with these
tools enlivens the reading process by allowing the reader to play,
experiment, and imagine new connections to the textual object.

\subsubsection{\emph{Voyant-Tools}}
\label{sec:org58fc9ad}
Jerome McGann "prosthetic extensions"
Potential texts: Woolf's \emph{Orlando}. 

\begin{itemize}
\item Interweave a narrative about touch. Taking new materialist ideas but
\end{itemize}
placing them within context of QPOC critiqe. Anzaldua and Bennet on
touch and severing. Sarah Ahmed too. 

\subsubsection{\emph{These Waves of Girls}}
\label{sec:org91edde3}
Following narrative desire. The click of the mouse allows readers to
move with the text, based on their own paths. 

\subsubsection{what are some print texts that enact these principles of movement?}
\label{sec:org76570d6}
\begin{itemize}
\item Alison Bechdel's "Are You My Mother": where every page is vibrating
\end{itemize}
with reference. 


\subsection{MISC}
\label{sec:orgcd3ee91}



\subsubsection{within our bodies}
\label{sec:org77846e9}
This point bears repeating---we are always stuck within the bodies of
our thinking. As such, we might as well turn to ourselves, to explore
(rather than how things are in the world) how things are \emph{to
us}. Sedgwick points out that the problem is not one of knowledge, but
one of movement. We can try to in-\emph{corporate}, as much as possible,
alternative reading methods that get at the unique experience of being
a thinking/feeling human that is fiddling with these tools. We can, in
other words, examine the possibilities of \emph{touching} what we read. And
we can do so with digital tools for text analysis and machine
learning.  However, there still exists a view that distant reading
lacks the sensitivity of close reading. "Critics who want to
sensitively describe the merits of a single work usually have no need
for statistics" (xxi).

\subsubsection{data reduction / queer assimilation}
\label{sec:orgddbdd40}
For those that would argue that negative feelings are no longer
relevant in today's world, Heather Love responds that the advent of
assimilation, of popular acceptance, only creates more problems for a
group that has come into being as abject. \emph{(the corrolary for digital
studies is the proliferation of data, of information, digitization)}
Queer assimilation and apparent rise in acceptance across popular
culture and mass media contradicts the reality of shame and stigma
that everyday queers experience, a contradiction that breeds ever more
shame: "Of course, same-sex desire is not as impossible as it used to
be; as a result, the survival of feelings such as shame, isolation,
and self-hatred into the post-Stonewall era is often the occasion for
further feelings of shame. The embarrassment of owning such feelings,
out of place as they are in a movement that takes pride as its
watchword, is acute" (4). What do we do with these residual feelings
of shame?

How should queer criticism orient itself? Love shows that critics face
a contradiction, brought on by the reality of negative feelings and
psychic costs of being queer in a homophobic society. The narrative
trajectory of queer progress runs counter to the residual pain of
being queer. Criticism is stuck in the middle of this ambivalence,
between affirming its pride and bemoaning its suffering: "We are not
sure if we should explore the link between homosexuality and loss, or
set about proving that it does not exist" (Love 3).

\subsubsection{Misc Quotes}
\label{sec:org0bd2dd9}
"how might activating emotion – leveraging it, rather than resisting
emotion in data visualization – help us learn, remember, and
communicate with data?" (Klein and D'Ignazio, \emph{Data Feminism},
chapter 2)

\section{Works Cited}
\label{sec:orga76b19f}
Caldwell, Kelly. "The Torment of Queer Literature," \emph{The Rumpus}. 2018.
Love, Heather. \emph{Feeling Backward: Loss and the Politics of Queer History}. 2009.
\end{document}
