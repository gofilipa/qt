% Created 2022-08-19 Fri 11:13
% Intended LaTeX compiler: pdflatex
\documentclass[11pt]{article}
\usepackage[utf8]{inputenc}
\usepackage[T1]{fontenc}
\usepackage{graphicx}
\usepackage{grffile}
\usepackage{longtable}
\usepackage{wrapfig}
\usepackage{rotating}
\usepackage[normalem]{ulem}
\usepackage{amsmath}
\usepackage{textcomp}
\usepackage{amssymb}
\usepackage{capt-of}
\usepackage{hyperref}
\author{Filipa  Calado}
\date{\today}
\title{}
\hypersetup{
 pdfauthor={Filipa  Calado},
 pdftitle={},
 pdfkeywords={},
 pdfsubject={},
 pdfcreator={Emacs 26.2 (Org mode 9.1.9)}, 
 pdflang={English}}
\begin{document}

\tableofcontents

\section{three}
\label{sec:org7ecfdc3}

\subsection{overview}
\label{sec:org803e1c5}

This chapter looks at the materiality of digital media by harnessing
concepts of the flesh as developed in black feminsist studies. 

First I consider how the flesh has been theorized in black feminist
studies, looking primarily at the concept of the Pornotrope, the
reduction of body to flesh, of the conceptual (body) to the material
(flesh). This reduction offers an analytic for understanding the kind
of capacious thinking required when two registers intersect---the
physical/analog and the immaterial/digital.

Through a discussion Octavia Butler's novel, \emph{Dawn}, I examine the
intersection between the digital and analog. In this novel, sexual
relations that occur through direct neural stimulation between alien
and human couples facilitated through a non-binary, intermediary being
with the ability to connect directly to the brain's pleasure centers.
This intermediary being who comes between the couples becomes a
substitute for flesh, an interface that dissolves the binary between
self and other, digital and analog, and offers a framework for
rethinking the role of agency in ethics.

Then, I examine a hypertext fiction work, a digital series by
Entropy8Zuper! entitled \emph{skinonskinonskin} (1999), and efforts at its
preservation amid the obsolescence of one of its major components,
Flash technology. I trace the complicated stack of hardware and
software that facilitate the display and preservation of this work,
examining various technologies that work in tandem such as internet
networks, hypertext, and Flash. These technologies together enable me
to explore how \emph{skinonskinonskin}'s interface engages the intersection
between the digital and analog registers undergirding its components. 


\subsection{intro: detail from Butler's \emph{Dawn}}
\label{sec:orgabd5ed5}
How can sex be more sensational if it bypasses the flesh? This is a
contradiction of registers, the analog and the digital. 

Explain a scene where they have sex, showing how it's like the analog
and the digital. (But the secret, which the chapter will reveal, is
that the digital is analog all long\ldots{})

Octavia Butler's novel, \emph{Dawn}, published in 1987, is the first book
of a trilogy, initially titled \emph{Xenogenesis} and more recently,
\emph{Lilith's Brood}, about humans who must, as a question of survival,
interbreed with extra terrestrial beings. \emph{Dawn} relates the story of
one family's matriarch, Lilith Iyapo, who is "rescued" from nuclear
war by the extra terrestrials, the Oankali. The story begins with
Lilith waking up in a prison cell on the Oankali spaceship, where she
soon finds out that the humanity has been all but extinguished and
that the Oankali intend to reproduce with the remaining survivors.

\subsection{flipping the pornotrope}
\label{sec:org3d7ae9e}
\subsubsection{outline}
\label{sec:orgae694a6}
\begin{itemize}
\item examining the contradictions
\item methodology to read into details
\item race emerges in the space of image, the floating space. the concept
of black flesh enable collision of two registers
\item reading escapability in the image, resisting the reduction to flesh (musser)
\item this is queer form (are there other people I want to look at?)
\end{itemize}


what do I need in order to talk about \emph{Dawn}? See my handwritten
notes.

\subsubsection{flipping the reductions: spillers to snorton, musser}
\label{sec:org460fa2f}
In the nearly impossible task of accounting of the egregious yet
overlooked of horrors of transatlantic slavery, black feminist
thinkers flip a concept that has been used to for the evils of
subjectification--the flesh. The idea of black flesh is a reduction of
the black body is first theorized by Hortense Spillers in her
influential essay, "Mama's Baby, Papa's Maybe: An American Grammar
Book." Here Spillers puts forth the conception of the black body as a
stack of "attentuated meanings, made in excess over time, assigned by
a particular historical order" (65). These meanings developed from the
body that had been reduced to flesh, a violent process, "severing of
the captive body from its motive will," that Spillers traces to the
middle passage and has four effects (67):
\begin{quote}
\begin{enumerate}
\item the captive body becomes the source of an irresistible, destructive
sensuality;
\item at the same time- in stunning contradiction - the captive body
reduces to a thing, becoming being for the captor;
\item in this absence from a subject position, the captured sexualities
provide a physical and biological expression of "otherness";
\item as a category of "otherness," the captive body translates into a
potential for pornotroping and embodies sheer physical
powerlessness that slides into a more general "powerlessness,"
resonating through various centers of human and social meaning. 67
\end{enumerate}
\end{quote}
Imposed by the reduction of bodies to bare physicality, to a material
substance, there is, in "stunning contradiction," some form
of meaning aheres to the flesh, a potential combination of sensuality,
objectificaiton, otherness, and powerlessness (Spillers 67).

Snorton suggests that this stripping (and imposition) of meaning
creates a possibility for resignifying black flesh, which "await[s]
whatever marvels of\ldots{} inventiveness" (Spillers 65). More recently,
black feminist theorists such as C. Riley Snorton, Amber Musser, and
Jayna Brown, for example, looks at the denigration of black flesh as a
simultaneous enabling of blackness as power. Snorton, for example,
demonstrates that the whittling down of black subjects to the
interchangeable, what he calls the "fungible," in chattel slavery
creates a possibility for for "fugitivity," or escape. His theorizing
uses the properties of flesh to trace the contradictions that produce
race and gender, for "Captive and divided flesh functions as malleable
matter for mediating and remaking sex and gender as matters of human
categorization and personal definition" (20). To demonstrate the
imbrication between sex, race, and black flesh, Snorton examines how
the denial of femininity of black slaves, whose bodies were used for
experimental purposes, enabled the emergence of field of gynecology as
a \emph{white} women's science. Snorton explains that, white femininity,
which prevents the inspection of white female genitalia, is
constructed out of the "scopic availability" of black flesh of women
(33). Yet, Snorton crucially asserts, this genderless flesh is also a
"capacitating structure" which enables "fungibility for fugitive
movement" (53). Here, Snorton interweaves various narratives of
fugitivity, such as that of Harriet Jacobs, who escaped slavery in
1842 and subsequently wrote \emph{Incidents in the Life of a Slave Girl}
(1861). Snorton points out that, while traditional racial "passing"
assumes white passing, it is blackness which gives one the "gender
indefiniteness" to enable "cross-gendered modes of escape" (56). The
"blackening" of Jacobs, for example, allowed her to obtain a level of
"fungibility, thingness" that precluded her recognition (Snorton 71).

For this "black and trans historiographical project," Snorton
demonstrates flesh as the physical and malleable material that enables
the construction of social categories like race and gender. Snorton's
method takes the details of the archival record as "conditions of
possibility" in his study, and reads for hints of resistance contained
within the dominant narrative that suggest the ommitted, obscured, or
elided experience of black subjects (Snorton 11). One example concerns
a slave woman, Anarcha, whose severe case of Vaginal Fistula made her
an object of medical experimentation. Because the record that hardly
mentions her by name (when it does not misname her) is silent about
her experience of what certainly were painful and prolonged surgeries
without anesthetics, Snorton turns to seemingly minor details like
smell and the presence of other enslaved women to "further imaginative
capacities" for reading resistance (14). Anarcha's reportedly
offensive smell, what Snorton calls the "discourse of disgust," for
example, indexes both her subjection to social hierarchy and her
resistance, for "the recurrent descriptors of Anarcha's body odor
provides an opening to imagine what modicums of protection might have
been afforded by smelling noxious to one's enslaver"
(27). Additionally, the banal reality of servitude implies that other
slave women such as Lucy and Betsey (among others who were unnamed),
stood as medical assitants when necessary, which opens speculation
into "modes of nourishment and care" they provided one another
(Snorton 29); In one procedure on Betsey, for example, Snorton
explains that "the opacity in the archive\ldots{} leaves room to imagine
how Betsey might have somehow resisted the performance of stoic
bravery or willing subjectivity that she was compelled to produce"
(25).

This methodology uses the critical imagination to read into what
Saidiya Hartman and others call "the violence of the archive," a
violence not only in the form of evidence, that the records literally
obscure or overlook information, but also in the tools of expression,
in the language that cannot approximate experience, and in the
discourse that dictates silence (2). Like "Critical fabulation,"
Snorton's reading of archival material draws narrative from the gaps
and lacunae of the evidence, working to "expose and exploit the
incommensurability between the experience of the enslaved and the
fictions of history" (Hartman 10). One difference however, which is
crucial for my purpose, is the influence of \emph{flesh} following Spillers
as the "zero degree of social conceptualization" (Spillers
67). Snorton's attention to flesh as a \emph{ground} for theorizing enables
him to attend to the implications of format and form in the
archive. For example, Snorton explains that the materiality of a
daguerreotype suggests "a visual grammar for reading the imbrications
of 'race' and 'gender' under captivity" (40):
\begin{quote}
To view a daguerreotype is to look at an image that does not sit on a
surface but appears to be floating in space. Rather than an antiquated
form of modern photography, as Foucault’s characterization implies,
the daguerreotype provides a series of lessons about power, and racial
power in particular, as a form in which an image takes on myriad
perspectives because of the interplay of light and dark, both in the
composition of the shot and in the play of light on the display. That
the image does not reside on the surface but floats in an unmappable
elsewhere offers an allegory for race as a procedure that exceeds the
logics of a bodily surface, occuring by way of flesh, a racial
mattering that appears through puncture in the form of a wound or
covered by skin and screened from view. 40
\end{quote}
The format of the dagguereotype evokes the method by which meaning is
stripped then reapplied to flesh that, for captive bodies, "functioned
as a disarticulation of human form from its anatomical features"
(18). The physical material of the image, that is the silvered copper
plate of the daguerreotype, at once solidifies its ground and indexes
an ambiguous space, what Snorton describes as the "unmappable
elsewhere" which swells to obscure while containing the evidence of
racial significations. Snorton's curious use of the word "puncture"
here recalls Roland Barthes's concept of the "punctum," which indexes
the experience of being pierced by a detail of the photograph (\emph{Camera
Lucida} 27). Opposed to the concept of the \emph{studium}, which represents
the dominant historical, social, or cultural meaning portrayed within
and by the photograph, the \emph{punctum} is the "sting, speck, cut, little
hole\ldots{} that accident which pricks me (but also buises me, is poignant
to me)" (Barthes \emph{Camera Lucida} 27-28). Barthes explains that,
"However lightning-like it may be, the \emph{punctum} has, more or less
potentially, the power of expansion. This power is often metonymic"
(\emph{Camera Lucida} 45). For Barthes, the \emph{punctum} is that detail of a
photograph which at once pierces the viewer and suggests an expansion,
an effect which is exagerrated in erotic photographs, where the
\emph{punctum}, "is a kind of subtle \emph{beyond}--as if the image launched
desire beyond what it permits us to see" (59). Barthes's theorization
of the \emph{punctum} allows us to see how the flesh can be at once a
\emph{mattering}, a becoming matter, and an accumulation of meaning, which
in simulteneity, has the effect peirces the viewer. We cannot, as
Snorton points out, locate the image at a specific point on the
copper-plate is essential, though we can feel its puncture. That the
image cannot be traced back to a single location, yet is contained and
signifies within that physical space, is crucial for undersanding the
way that the physical registers interact with symbolic ones. The
meeting between this liminal space of the image's meanings and its
silver-plated copper ground offers a means for thinking through the
collision of two registers, of flesh against gender \& race.

Amber Jamilla Musser runs with the notion of unmappability to theorize
surface aesthetics in her conception of \emph{brown jouissance}. Like
Snorton, Musser develops her ideas alongside a reading of photograph
technology, in this case, from a photograph of the artist Lyle
Harris's "citation" of Billie Holiday. Musser's reading of the
photograph theorizes the way that escapability emerges from the
superfical, the surface of this image, in its "shine" revealing a self
that is excessive yet inaccessible. Musser notes the details of
Harris's dress, such as the "pearls, eye shadow and lipstick" that
capture the light of the image, as the "Shine [which] plays joyfully
with the idea of the body as body while rejecting the demand to
present anything other than surface" ("Surface-Becoming"
par. 3). Moving beyond the unmappable elsewhere, surface for Musser
not only indexes but \emph{forecloses} access to that which we cannot know:
She explains, "we can understand surface as the underside of the
scientific/pornographic drive toward locating knowledge in an
'objective' image" ("Surface-Becoming" par. 2). The material of the
surface, which "highlights the mutability of the flesh rather than
interiority," suggests that which escapes its own form (Musser,
"Surface-Becoming" par. 11). While foreclosing the reading of
interiority, the surface aesthetics opens up other considerations,
such as the evocation of hunger in Lyle's open mouth: "sticking with
the surface--–through the use of Polaroid and shine–--illuminates the
possibilities of imaging a self whose interiority we do not have
plumb, but who still generates creative possibilities for resisting
the mandate of sovereign subjectivity" ("Surface-Becoming"
par. 23). To locate in something physical that which eludes
physicality is a way of resisting (without attempting to recuperate)
the reduction of body to flesh.

\subsubsection{{\bfseries\sffamily TODO} via the pornotrope, the flesh becomes a queer form}
\label{sec:orgaa7a31f}
This method is about \emph{taking what has been a method of reduction},
what has been a tool for appropriating the complexity of real world
objects for the purpose of exploitation, and using that \emph{to instead
seek out moments of illegibility}, opacity. To multiply rather than
resolve meaning.

\subsection{alien sex (is queer bc sensational, bypassing flesh?)}
\label{sec:org7f0edd3}
\subsubsection{outline}
\label{sec:org78d7069}
\begin{itemize}
\item Butler establishes fear and violence as innate to humanity.
\item Alien sex in \emph{Dawn} creates new interface that dissolves binaries
between digital/analog and agency.
\item Brown, Jackson, other criticism about Butler and \emph{Dawn}.
\item Pose black feminist thinkers on the question of sensation against
traditional white male accounts. Counter William James with Dariek
Scott.
\end{itemize}

\subsubsection{\emph{Dawn}}
\label{sec:orga6599e6}
\begin{enumerate}
\item For Spillers, Snorton, and Musser, the flesh has been a location from
\label{sec:org270b247}
which to theorize the various racial and gendered meanings ascribed to
it over time. The physical materiality of the flesh, which indexes a
liminal space where meaning is simultaneously ascribed and obscured,
functions a solid ground for thinking through the paradoxes of
mediation through touch and sensation. In \emph{Dawn} specifically, the
human/alien couplings allow us to think through the ways that flesh,
as a medium for touching, emotional, and physical connections, at once
enables and prevents interpersonal unity in same- and cross-species
couplings.

In \emph{Dawn}, the Oankali effectively force Lilith to do what she
describes as an impossible task: to shepherd a group of humans into
first understanding, then accepting, their participation in the gene
trade. Like the "Judas goat" who leads the sheep to slaughter, Lilith
must convince the humans to accept a fate whose implications she
cannot bear: that humanity will become something profoundly different
from what it is; that their children will look not quite human, what
but like "Medusa children" (Butler 87). Lilith's aversion to the
prospect of humanity changing signals a deep predicament which she and
other humans will struggle against for the rest of the series. As
Jayna Brown points out, much of Bulter's work explores the paradox
that humanity must fundamentally change in order to survive. Speaking
of the \emph{Parable} novels, which also depict the events following
societal collapse, Brown explains that, for humanity, "changing and
prevailing cannot coexist\ldots{} We must adapt to survive, but species are
never stable over time if they successfully adapt" (Brown
94). Throughout the novel, Lilith toes the line of this paradox,
simultaneously encouraging the humans to obey the aliens' directions
while preparing them for escape, encouraging them to "learn and run."
Nonetheless, for more than a few humans in her group, she represents
acquiesence to their new captors, and becomes a recepticle for their
anger and resistance to colonization. Despite the Oankali's promise of
long, healthy lives for their human mates, it is difficult for the
humans to take their advances as promising anything but annihilation.

One reason that humans cannot accept the gene trade can be attributed
to their own genetics which contain, in the words of the Oankali
themselves, the "human contradiction." As explained by Jhdaya, the
first Oankali that Lilith meets, humans have two characteristics that
once enabled humanitity to survive but now threaten the future of the
species:
\begin{quote}
"You are intelligent," he said. "That's the newer of the two
characteristics, and the one you might have put to work to save
yourselves. You are potentially one of the most intelligent species
we've found, though your focus is different from ours. Still, you had
a good start in the life sciences, and even in genetics."
"What's the second characteristic?  
"You are hierarchical. That's the older and more entrenched
characteristic. We saw it in your closest animal relatives and in your
most distant ones. It's a terrestrial characteristic. When human
intelligence served it instead of guiding it, when human intelligence
did not even acknowledge it as a problem, but took pride in it or did
not notice it at all. . ." [\ldots{}] "That was like ignoring cancer. I
think your people did not realize what a dangerous thing they were
doing."
\end{quote}
The tendency toward hierarchy, as a "terrestrial" characteristic, is
ingrained in all humans. The impulse to stratify people, to create
social groupings, even to colonize and oppress, descends from an
ancient instinct that once served to sustain, protect, and organize
early human tribes. But when the hierarchical instinct grows unchecked
into the modern world, Jdhaya explains, it becomes dangerous, like a
cancer. The problem, which the book returns to again and again, is
that humanity will not overcome something which is ingrained into
their DNA. Brown points out that, "If survival is based on a
competitive struggle for existence, then how can humans ever overcome
what they are biologically wired to do?  How can humans change \emph{and}
stay the same?"  (Brown 98-99).

The problems that emerge from this hierarchical impulse are those that
precipitate and perpetuate systemic inequalities. Stratifications
between gender, race, nationality, and class, for example, descend
from this very foundational tendency to mark and divide what is
different, what is other, from what is familiar. This novel explores
how such a tendency, deeply ingrained in human nature, comes to the
fore even as it is threatened by aliens who intend to "fix" the human
contradiction through gene manipulation in interbreeding. The tendency
to self-organize appears early on, when the humans are being woken up
from suspended animation in order to prepare and train for
survival. Upon waking, the humans almost instantly pair off in
male-female couples that end up sustaining themselves throughout the
story. The pressure to couple brings a remnant of human society into
the strange, alien spaceship which offers some form of stability for
the survivors. When one woman hesitates to choose a man, a violent
confrontation ensues. One of the women involved demands: "What the
hell is she saving herself for?\ldots{} It's her duty to get together with
someone. There aren't that many of us left" (335). The stratifications
that the hierarchical tendency engenders grows as the humans become
more desperate in their resistance against the Oankali
colonization. In particular, certain men, like Peter and Curt for
example, grow more and more agitated at their captivity and eventually
attack Lilith and her followers, who they regard as responsible, with
deadly force. 

\item Crucially, this novel refrains from offering a solution to the
\label{sec:orge9bdbcb}
problems that accompany the characteristic of hierarchy. Rather, it
explores the implications of humanity's hierarchical nature for
reconsidering the role of agency in relationships and ethics. [INSERT
MUSSER QUOTE ABOUT VIOLENCE OFTEN BEING SOMETHING WE MUST COEXIST
WITH]. The book offers ways of re-conceptualizing relationships
without fixing or recuperating histories of violence and oppression.

\item Early in the story, the fear of the unknown is established as a
\label{sec:org9a65c76}
stubborn and innate human trait that results in using familiar
categories and concepts for encountering new phenomena. The first time
that Lilith comes face-to-face with her captors, she processes the
alien body in human terms:
\begin{quote}
The lights brightened as she had supposed they would, and what had
seemed to be a tall, slender man was still humanoid, but it had no
nose--no bulge, no nostrils--just flat, gray skin. It was gray all
over--pale gray skin, darker gray hair on its head that grew down
around its eyes and ears and at its throat. There was so much hair
across the eyes that she wondered how the creature could see. The
long, profuse ear hair seemed to grow out of the ears as well as
around them. Above, it joined the eye hair, and below and behind, it
joined the head hair. The island of throat hair seemed to move
slightly, and it occurred to her that that might be where the creature
breathed--a kind of natural tracheostomy.

Lilith glanced at the humanoid body, wondering how humanlike it really
was. "I don't mean any offense," she said, "but are you male or
female?"

"It's wrong to assume that I must be a sex you're familiar with," it
said, "but as it happens, I'm male."

Good. It could become 'he' again. Less awkward. 29
\end{quote}
That she initially describes his features by cataloging a "nose,"
"hair," "eyes," "ears," and "throat," when he has no such organs,
illustrates the strength of the impluse to interpret bodily
appearances according to anatomical categories that have been
inculcated. Then, the first question that she asks him--about his
sex--shows the priority that gender addressibility has for ascribing
the alien's presence and being. Although Jdhaya points out Lilith's
oversight, that she can address him on male terms offers some comfort
to the situation. When, however, the strangeness of the alien's
appearance proves beyond the terms available to her, she immediatley
turns to fear:
\begin{quote}
She did not want to be any closer to him. She had not known what held
her back before. Now she was certain it was his alienness, his
difference, his literal unearthliness. She found herself still unable
to take even one more step toward him.

"Oh god," she whispered. And the hair--the
whatever--it-was--moved. Some of it seemed to blow toward her as
though in a wind, though there was no stirring of air in the room.

She frowned, strained to see, to understand. Then, abruptly, she did
understand. She backed away, scrambled around the bed and to the far
wall. When she could go no farther, she stood against the wall,
staring at him.

Medusa. 30
\end{quote}
Interestingly, when the reality of his alien appearance does begin to
register, it occurs in a pre-linguistic, instinctual level. First, she
describes an intense aversion toward physical proximity--"She found
herself unable to take even one more step toward him" (29-30). Once
she takes a closer look at his "features," she still struggles to
process his physical composition, and when she
"abruptly\ldots{} understand[s]," her impulse is to move away, an aversion
that is described through terms of body language. When the narration
finally addresses the alien's appearance directly, it uses figuration,
an evocation of the mythical figure "Medusa," which subscribes what
she sees to something that is, ableit a fantastical creature, already
known to the human imaginary. She cannot describe in plain language
what is beyond her registers of familiarity; she must turn to
figuration. The narration of body movements and figuration in this
passage reinforces the processes that humans undergo when encountering
the unknown: when the attempt to apply existing experience and
knowledge, such as hierarchical structures, fails, the outcome is
pure fear.

The way that Lilith handles her first encounter with alien life
displays the workings of the human contradiction. The tendency toward
hierarchy demands that she place this being on a scale of familiarity,
comparing him to what she already knows about other living beings, for
example, that Jhadaya is male. However, the hierarchy fails to subsume
his other qualities, the sensory organs, which are beyond what Lilith
has witnessed in a living being. Then, intelligence steps in to create
an analogy, in this case to Medusa, and her mind makes the leap
between what she sees and what she already knows. The two sides of the
contradiction, hierarchy and intelligence, work together here to
engender a sense of all-consuming fear of the other. 

\item acquisitiveness leads to collectivity, refiguring ethics
\label{sec:org5139944}

Like the humans, the Oankali also have innate tendencies that
determine the way they engage with new and unfamiliar species. For the
Oankali, ther deepest instinct is to acquire new species for their
"gene trade." Jhadaya explains: "We acquire new life, seek it,
investigate it, manipulate it, sort it, use it. We carry the drive to
do this in a minuscule cell within a cell, a tiny organelle within
every cell of our bodies" (84). The Oankali compulsion to acquire may
seem to have some similarities with the human drive for hierarchy, in
particular, that it requires taking in and incorporating new beings
into an existing structure. However, there is a crucial difference
between the Oankali and the humans, which has to do with the
collective nature of the alien species. One of the Oankali children,
Nikanj, explains to Lilith that they evolved from a life form that
consisted of numerous interconnected beings: "'Six divisions ago, on a
white-sun water world, we lived in great shallow oceans,' it said. 'We
were many-bodied and spoke with body lights and color patterns among
ourself and among ourselves'" (123). From their "many-bodied"
ancestors, the current Oankali inheirited a constitution of
collective, rather than individual, consciousness, which affects their
concept of agency. As the Oankali evolved, this collective nature
affects the way they communicate, which is by sharing sensory
information directly so that the interlocutor experiences what is
being related to them, and the way they make decisions, which is by
unanimous agreement. Agency is distributed among the beings, who are
singular and plural at once, "ourself and ourselves."

Their method of acquisition, though arguably similar to human acts of
colonization in the way they expand through incorporation, presents a
drastically different understanding of ethical relationships. Because
health and vitality are necessary in order to trade genes, the Oankali
do not admit any form of harm or desctruction of life. At several
points in the book, this attachment to life becomes a blind spot,
preventing them from anticipating acts of violence and at one point,
even death, by humans. Toward the end of the novel, Lilith's partner,
Joseph, is killed by a group of humans who rebel and attempt to escape
the Oankali. Soon after his murder, Nikanj uses Joseph's genetic
material to impregnate Lilith without her knowledge. Nikanj explains
to Lilith that it gives her what she truly wants, though she cannot
admit it,
\begin{quote}
"You'll have a daughter," it said. "And you are ready to be her
mother. You could never have said so. Just as Joseph could never have
invited me into his bed--no matter how much he wanted me
there. Nothing about you but your words reject this child." 468-9.
\end{quote}
Nikanj's explanation here reformulates the question of agency from
conscious acknowledgement ("You could never have said so") into
subconcious desire. For the Oankali, pleasure is the principal factor
for decision-making. For, unlike humans, Oankali lack the capacity to
self-delude through language. When Lilith protests that "It won't be
human," Nikanj points out the facts: "You shouldn't begin to lie to
yourself. It's a deadly habit. The child will be yours and Joseph's"
(469). Nikanj's reasoning is simple: no matter what she says, it knows
she will love and accept the child. The collective nature of the
Oankali prioritizes pleasure, in this case pleasure tied both to
concieving a child and mothering a child, over what the organism says
or thinks. The question of agency, and therefore what counts as
coercion and manipulation, depends on pleasure rather than individual
choice, which has significant implications for ethics.

\item Alien sex paradoxically bypasses flesh to amplify sensation
\label{sec:orgd550407}
One of the consequnces of the Oankali focus on pleasure as a
foundation for ethics may seem perhaps paradoxical from a human
perspective. The method that Oankali have sex dispenses with what is
for humans the central enabler of pleasure--the flesh. Flesh, which is
the interface through which humans achieve sexual contact, is an
obstacle for Oankali sex. In the Oankali sexual union, the male and
female do not touch, but are rather connected through an intermediary,
nonbinary being whose "sensory arms" plug directly into into the brain
through the back of the neck. The ooloi intermediary dispenses not
only with the clumsiness of human bodies and the flesh, but also with
human modes of communication and intimacy, to achieve direct
stimulation of the brain's pleasure centers. As it seduces Joseph,
Nikanj explains that he "offer[s] a oneness that your people strive
for, dream of, but can't truly attain alone" (359). The direct
connection facilitated by the ooloi between human couples offers a
sensual and cognitive experience which cannot be paralleled by
physical intercourse. The following passage which relates Lilith's
experience when "plugged into" Joseph, merits quoting in full. After
Nikanj connects to Lilith and Joseph, she
\begin{quote}
immediately recieved Joseph as a blanket of warmth and security, a
compelling, steadying presence. 

She never knew whether she was receiving Nikanj's approximation of
Joseph, a true transmission of what Joseph was feeling, some
combination of truth and approximation, or just a pleasant fiction. 

What was Joseph feeling from her?

It seemed to her that she had always been with him. She had no
sensation of shifting gears, no "time alone" to contrast with the
present "time together." He had always been there, part of her,
essential. 

Nikanj focused on the intensity of their attraction, their union. It
left Lilith no other sensation. It seemed, itself, to vanish. She
sensed only Joseph, felt that he was aware only of her. 

Now their delight in one another ignited and burned. They moved
together, sustaining an impossible intensity, both of them tireless,
perfectly matched, ablaze in sensation, lost in one another. 308-309
\end{quote}
There are a couple key qualities about human sexual relations that can
be gleaned from this passage. First, and significantly, that Lilith
questions whether her mental experiences are "real" or not, the fact
that she doubts, points to an issue with human intimacy--that there is
space for mistake and miscommunication. Perhaps the main reason that
the Oankali sexual experience is so compelling for humans is that it
eliminates the potential for miscommunication and clumsiness that
occurs between partners when desire is expressed through the
flesh. Which leads to the second point, the relationship and
interworkings of physical sensation and mental images. Because
physical contact only occurs in the brain, it is easy to assume, as
Joseph eventually does, that the experience is a simulation or
"illusion," in his words. However, as Nikanj explains,
"Electrochemical stimulation of certain nerves" achieves the same
effect as skin to skin contact, only replacing the intermediary of
flesh with a direct neural connection (359). Nikanj continues, "What
happened was real. Your body knows how real it was. Your
interpretations were illusions. The sensations were entirely real"
(359).

\item {\bfseries\sffamily TODO} The pleasure, images received, are the "unmappable elsewhere"
\label{sec:org8c9686a}
based on the flesh. "interpretation" quote

doubling of flesh and pleasure center. 


Pleasure forms the foundation for Oankali ethics. As a result, in what
is perhaps a paradox from a human perspective, the flesh becomes both
an obstacle and an enabler of sexual contact. In the Oankali version
of sex, union between a male and a female is faciliated through an
ooloi, a non-binary, intermediary. The ooloi mediates sex by


That the ooloi substiutes flesh has two implications
1 - pleasure: where is pleasure located, in the interface that fades
  from view
\begin{itemize}
\item The sharing of sensory experiences via the central nervous
system.  This is 'jacked in'. We see this in Neuromancer. The
idea that a human can directly link or connect to another
system. It's a hallucination.
\item What does it say about pleasure and touch? About pleasure and
mediation? The question of locating pleasure in the \textbf{interface},
of whether we can locate pleasure in the immediacy of an
interface that fades from view---from the direct 'connection'
between two neural systems---or from the levels of mediation
between two entities, such as the skin, sexual organs, etc.
\end{itemize}
   -> breakdown of binary/analog: Even the alien sex is analog: at the
      deepest level, it is about feelings and sensations. It is about
      feeling like you're being touched.  
2 - Rethinking agency toward pleasure. See philosophy of perception.

\item Philosophy of Perception
\label{sec:org4cf2ea2}

Butler is resituating ideas about Embodied Consciousness within a
feminist/radical ethics. How our thoughts and feelings are actually
based in the body. Which changes the way that we think about pleasure,
which changes the way we think about ethics. 

\item intelligence will handicap itself -> see history of male
\label{sec:orgdd5c421}
theorizing on the sensation which overlooks what it well pleases

It's interesting that reason will handicap itself from diagnosing its
own irrationalities. It will explain them away. Why does reason do
this?  Intelligence is largely an operation that orders, organizes,
things. Makes judgements and therefore places things in a
hierarchy. This organization or ordering is inherent to the way that
intelligence works, right? So there is the conflict. Reason can
rationalize hierarchy as a good thing, as an efficient thing.
\end{enumerate}


\subsection{\emph{skinonskinonskin} (1999)}
\label{sec:org4f0342e}

\subsubsection{adobe flash}
\label{sec:orgb866fc7}

\section{Works}
\label{sec:orgdd62b33}
Barthes, Roland. \emph{Camera Lucida}.

Hartman, Saidiya. "Venus in Two Acts." \emph{Small Axe}, vol. 12 no. 2,
   2008, p. 1-14. Project MUSE muse.jhu.edu/article/241115.

Musser, Amber Jamilla. "Surface-Becoming: Lyle Ashton Harris and Brown
  Jouissance." \emph{Women \& Performance}, vol. 28,. no. 1. February 26, 2018
  \url{https://www.womenandperformance.org/bonus-articles-1/28-1-harris}. 

Snorton, C. Riley. \emph{Black on Both Sides}

Spillers, Hortense. "Mama's\ldots{}"
\end{document}
