% Created 2023-07-16 Sun 16:57
% Intended LaTeX compiler: pdflatex
\documentclass[11pt]{article}
\usepackage[utf8]{inputenc}
\usepackage[T1]{fontenc}
\usepackage{graphicx}
\usepackage{grffile}
\usepackage{longtable}
\usepackage{wrapfig}
\usepackage{rotating}
\usepackage[normalem]{ulem}
\usepackage{amsmath}
\usepackage{textcomp}
\usepackage{amssymb}
\usepackage{capt-of}
\usepackage{hyperref}
\author{Filipa  Calado}
\date{\today}
\title{}
\hypersetup{
 pdfauthor={Filipa  Calado},
 pdftitle={},
 pdfkeywords={},
 pdfsubject={},
 pdfcreator={Emacs 26.2 (Org mode 9.1.9)}, 
 pdflang={English}}
\begin{document}

\tableofcontents

\section{intro}
\label{sec:orgc160c89}

\href{file:///images/desire.png}{The desire to write is the desire to fool you,
seduce you. Here I am - again - always getting the girl, saying the
right thing or (toss this in for effect) something deliciously,
winsomely wrong. Look over there - that's me, at four\ldots{}}

Caitlin Fisher's electronic work, \emph{These Waves of Girls}, tempts the
reader through its pages with links to seductive phrases like "the
desire" or "that's me, at four\ldots{}"  ("desire\(_{\text{to}}\)\(_{\text{write.htm}}\)"). Though
they are associative in nature, supporting the work's de-centralized
formal structure, the links also defer narrative coherence within the
story, a sexual coming of age of a young lesbian woman named
Tracey. Clicking through a few episodes, the text reveals itself to
be, like much of the "hypertext fiction" of the time,
disorienting. The links offer a profusion of narrative paths that
disrupt the relationship between cause and effect, halting the
narrative progression in what James Pope describes as "a baffling
range of choices for movement which actually led to a stifling of
movement altogether" (11).

\href{file:///videos/erotic.gif}{Walkthrough of part of the work from opening menu}

The difficulty in "grasping" the story on a physical level, in
clicking through the various episodes toward a continually deferred
sense of narrative closure, mirrors the cycles of desire and
frustration within the story world. In addition to dogging this
deferred story arc, the reader also follows the narrator's elusive
promise of sexual discovery, of which she only ever gets part. One
highly eroticized scene, for example, uses linking to cast the
referent of Tracey's desire into doubt. Here, the 15-year-old Tracey
performs a gymnastics "beam routine" for an older man:
\begin{quote}
‘I don’t want to have sex,’ I say. ‘Not with you.’

But then, not wanting to disappoint, ‘I could do my beam routine.’  

I take off my clothes and, fifteen, mount the line of the carpet to
perform my entire junior beam routine, handstand press, two
backhandsprings included. He jerks off. Dismount.  

I hurry into my clothes and head home to Vivian (who loves me, but not
as much as I love her). For years I worry that the shoe salesman was
really disappointed, stuck with a fifteen year old virgin gymnast
rather than a real bad girl. "Beamroutine8.htm"
\end{quote}
The beam routine performance, while compensating for the sexual act
that Tracey withholds from the man, also suggests a desire that is
inaccessible to touch. The inaccessibility operates on two levels:
narrative and formal. On the narrative level, the referent of the
sentence, "I don't want to have sex\ldots{} Not with you" is cast into
doubt by its structural relationship to the rest of this episode, in
which Tracey declares her love to her friend, Vivian, with whom she is
picking up men at the bar: "Our third daiquiris arrive and, my hand
hot on her waist, "I tell her I love her. She loves me too, she says,
just not as much as I love her" (“beamroutine5.htm”). The instability
of desire's referent, which operates on the level of narrative, also
works through the text's associative format, which seduces and draws
the reader to click through to new parts of the story that frustrate
its coherence. The link to this same sentence, "I don't want to have
sex," literalizes desire as an ever moving target. It leads to an
often-linked node about another love object, Jennie Winchester:
\begin{source}
I’m in bed with Jennie Winchester and I realize she wants me to undo
her pants. She needs to be home by 11:00 and needs to leave my place
by 10:45. I’m kissing her but opening my eyes at intervals to catch
the clock. At exactly 10:43 I unbutton her Levis and shove my hand
inside, barely undoing the zipper. “I’m in bed…”
\end{source}
The connection between Jennie, who is depicted as a highly desirable
sexual partner throughout the work, and the phrasee "I don't want to
have sex," casts Tracey's desire into doubt. These connections
reinforce a fundamentally disorienting effect of the links, in which
the reader loses context as she familiarizes herself with the events
of the story. In facilitating this narrative elusiveness, the links
thus reinforce a frustrated relationship to touch across the work,
where the reader clicks through a narrative to encounter a story she
will never fully grasp, to read a story about a desire that is never
fully satisfied. 

\subsection{intervention}
\label{sec:org7d525f8}
I open with this work because it illustrates a foundational assumption
for this disseration--that digital media abstracts complex elements of
queer identity, desire, and experience in text into formal
possibilities for analysis. In \emph{Waves}, for example, desire is
distributed across a network structure, and the thus network becomes a
formal device for exploring the workings of queer desire. This text
puts into practice post-structuralist principles about the
destabilization and decentering literary narrative, as George Landow
points out in his seminal work on hypertext theory that draw from the
ideas of Roland Barthes and Jacques Derrida.\footnote{According to Landow, post-structuralists like Derrida and
Barthes offer a helpful approach to textuality for understanding the
affordances of hypermedia. For example, they destabilization of ideas
of center, margin, hierarchy, and linearity, and their replacement
with ones of multilinearity, nodes, links and networks. Landow,
George. \emph{Hypertext 3.0: Critical Theory and New Media in an Era of
Globalization}. 2006. Print.} In \emph{Waves}, as
readers move through a web or network of texts, the center of the
story continually shifts and scrambles the reader's sense of
direction. Despite this apparent intractibility, this movement is
highly constrained. Within the network, according to Alexander
Galloway, "control is synonymous with possibility."\footnote{Galloway explains that, although "protocol," or network
controls, is central to the way that networks operate (in order to be
part of the network, one need to subscribe to the network’s "rules" or
structure), protocol also creates an opening, a possibility (an
exploit, vulnerability), which is about harnessing what’s inherent in
the network structure.} Movement
requires a shared standard, language, or what Galloway calls
"protocol." Operating mostly out of view, this protocol constrains the
possible connections between pages in the story.

My work examines digital media for such constraints, particularly
those that abstract elements of sex, sexuality, and gender into
digital forms. Then, I explore how these constraints might engage
theories and concepts from the field of Queer Studies. The network
story format as a controlling structure, for example, shares in the
same principles that also influence the inception of Queer Studies
with Judith Butler's \emph{Gender Trouble: Feminism and the Subversion of
Identity} in 1990. Butler's work establishes a denaturalizatin of sex,
gender, and sexuality as a foundational assumption for all subsequent
Queer Theory. Butler, who draws from Michel Foucault's examination of
power and its operation through "discourse,"\footnote{In particular, Foucault theorizes the repression the discourse
of sexuality though through codifications and multiplications of
sexuality.} famously exposes
these categories as social constructions that precede and determine
subjectivity rather than express it. Her work lays the groundwork for
Queer Studies's fixation with seeking out exploits that make
resistance possible within the matrix of power. Bringing this approach
to \emph{Waves}, I would explore how network dynamics, in which control is
central to the way that networks operate, engage with social power
structures that determine and control sexuality. If discourse, like
protocol, determines all connections, then the story/subject only
comes into intelligibility through the matrix of control. How does
this constraint relate to queer identity as a controlling structure?
When computation and coding start to determine how we engage with
texts, what kinds of analysis or experience do they make possible?
This is a question I will return to at the end of this introduction.

\subsection{"Queer DH"}
\label{sec:orgfa97816}
This project engages the emerging field of \emph{Queer DH}, which
intersects two seemingly incommensurable fields—-the Digital
Humanities (DH) and Queer Studies. While much of DH scholarship is
enticed by the proposed objectivity of digital methods, which borrow
from the social sciences methodologies to verify, correct, or
establish knowledge, Queer DH emphasizes instead how digital tools and
methods might productively complicate the legibility and stability of
knowledge. To do so, Queer DH takes one of two general approaches: the
first approach aims to critique problematic aspects of computational
processes (such as the way that they can reduce the complexity of
data) by imagining alternative ones, while the second approach
maintains that queerness is inherent to computational logic. 

The first approach often consists of speculative or critical making
projects that do not work in a functional sense. Rather, these
projects problematize the constructed nature of technical objects. For
example, Zach Blas and micha cárdenas’s computer program,
\emph{transCoder}, contains functions inspired by Queer Theory, such as the
"destabilizationLoop," which "breaks apart any process that acts as a
continuously iterating power" ("libraries.txt"). Below is a list of
functions contained within the program:
\begin{SOURCE}
exe()
will literally execute the semantic meaning, regardless of the computing logic

iDo()
computer will self-destruct

theCloset
variable that will silence whatever it is assigned to

\ldots{}

nonteleo()
strips any program of a goal-oriented result

fistFuck()
generates never-ending feedback loop
\end{SOURCE}
The above pseudo-code explores how a "cultural layer of
queerness\ldots{} acts upon and mutates mutually with the computer layer of
algorithms, binary logic, data structures, code, software, and
digitization" ("about.txt"). Another example of this speculative
strand of Queer DH is a project called "Queer OS," which posits a
theoretical operating system that puts into play concepts from queer
theory. It's creators draw from Kara Keeling's influential article
from \emph{Cinema Journal}, where Keeling coins "Queer OS" as a "scholarly
political project" that "take[s] historical, sociocultural, conceptual
phenomena that currently shape our realities in deep and profound
ways, such as race, gender, class, citizenship, and ability\ldots{} to be
mutually constitutive with sexuality and with media and information
technologies" (Keeling "Queer OS"). The operating system proposed by
Queer OS's creators includes an interface that "disappears but is not
naturalized," and "promiscuous" applications that "move and interact
across platforms, devices, users, and geographical regions
unrestricted" (Barnett et al). Such projects imagine computational
tools that "[do] not yet exist and may never come to exist [… do] not
yet function and may never function" (Barnett et al). They explain
that the motivation behind such projects goes beyond the impulse to
deconstruct current systems:
\begin{quote}
While our queer impulse may be to explode this [black] box, to lay
bare its inner workings in a gesture of radical revelation, this
desire to access the truth of the machine in that hardware, those
circuits, these gates and switches is rooted in a drive toward depth,
essence, and resolution that is antithetical to a QueerOS\ldots{} a more
productive interface would be expansive, proliferating the
relationality allowed for by the inter-face, its inter-activity, its
nature as that which is between or among, that which binds together,
mutually or reciprocally. "Interface"
\end{quote}
Such speculative projects share a suspicion that the "drive toward
depth" in deconstructive analysis undermines a queer ethos, which is
playful, provisional, and cannot be captured by exposure.

In contrast to this speculative approach, one strand of Queer DH
explores how technological systems and tools already engage queer
concepts or methods of analysis. For example, work by Jacob Gaboury
explores how "NULL" values, which denote values that are unknown,
evokes a distinctly queer logic, a "refusal to cohere, to become
legible" (“Becoming NULL”). In database computing, The NULL value
stands in place for a value that is missing, but that cannot be
equated to zero, which itself represents a quanitity of nothing. By
"surfac[ing] a queer technics lying at the heart of the database
form," the NULL value identifies a presence or existence without
giving any more information away (“Becoming NULL”). According to
Gaboury, this value enacts a "retreat from representation sits at the
heart of queerness" ("Becoming NULL"). Moving from the database to
data formats, Textual scholar Julia Flanders explores the TEI (Text
Encoding Initiative), a "markup language" for adding descriptive tags
to textual data. Flanders explores what she calls the "queerability"
built into this data structure, an ordered hierarchy of elements with
strict naming conventions and boundaries around elements. Despite the
rigidity of the TEI structure in which elements must be bounded and
discrete, Flanders argues that it offers a possibility for dissent,
for expressing smooth information, through the customization and
nesting of elements. According to scholars like Gaboury and Flanders,
it is from within the structuring logics of computer systems
themselves that queerness finds the space to operate.

Cutting between these approaches, this project first takes a
deconstructive look at digital tools to seek out a constraint, that
is, an aspect about the tool which reduces or collapses the nuances of
queer identity and experience into computable components. Then, I
experiment with how this reductive aspect might be re-worked to bring
back the details of queer identity and experience. Each chapter of my
dissertation takes up a different computational constraint, such as
data formats or programming logics, and explores how these can be
re-deployed to reflect the multiplicity, fluidity, and dynamicity of
queerness as it has been theorized by Queer Studies. Here I take the
necessary disambiguation of computational processes, which fix and
categorize literary text into static and legible data, toward
revealing the complex and ambiguous forms of queer identity and
experience. For example, in my second chapter on the TEI, or "text
encoding,", I take the rigid and hierarchical format of text encoding
framework as an opportunity to think productively about hierarchical
power structures.  In the careful and minute work of this encoding
process, where each element of a text is tagged within a document
hierarchy, the editor grapples against the limitations of the
compulsory categorization and containerization of data. To work within
the limitations of the document hierarchy, I borrow from Queer of
Color Critique and the scholarship it has inspired around the archive
of slavery. Here, I take historiographical methods such as Saidiya
Hartman’s "critical fabulation," or the histories of what could have
been, but that do not fit into dominant systems of knowledge, as a
critical strategy for working within hierarchical structures of
dominance. This close and careful work with text encoding allows me to
make the connection between hierarchical data structures and power
dynamics, offering an opportunity for rethinking my usage of the text
encoding tool to foreground that which has been excluded from the
system.
\subsection{computational constraints}
\label{sec:orgde14334}
In order to seek out computational constraints, this work unpacks two
common assumptions about software and data. The first assumption is
that software is neutral. Here, Tara McPherson's work on the history
of operating systems shows show hows software development encodes
social ideology, specifically absorbing hegemonic assumptions about
how to handle and organize difference. Examining the technological and
social moment of the 1960s United States, McPherson poses the birth of
the operating system (OS), the foundational software that supports a
computer's basic functioning, alongside emerging discourses on racial
equality. She surfaces how "the organization of information and
capital" in OS development resonates in the neoliberalist discourses
that "distanc[ed] the overt racism of the past even as they contained
and cordoned off progressive radicalism" (30). She points out how
UNIX-based systems, which partition and simplify complex processes
into discrete components, evoke the ways that identity politics
cordones off social groups into manageable units. While this
organization is productive for the promotion of civil rights, it also,
according to McPherson, "Certain modes of racial visibility and
knowing coincide or dovetail with specific ways of organizing data"
(24). She offers an example with the "rules" of UNIX philosophy:
\begin{quote}
Rule of Simplicity: Design for simplicity; add complexity only where
you must. 

Rule of Parsimony: Write a big program only when it is clear by
demonstration that nothing else will do. 

Rule of Transparency: Design for visibility to make inspection and
debugging easier\ldots{} 

qRule of Representation: Fold knowledge into data
so program logic can be stupid and robust. 26
\end{quote}
According to McPherson, these rules correspond to ideological values
for partitioning and organizing difference. The rules of "Simplicity"
and "Parsimony," for example ensure that programs will be composed of
small, interlocking parts that can be easily updated and transported
to newer versions. And the rules of "Transparency" and
"Representation," flatten nuance, ambiguity, and "raw" data into
legibility. According to McPherson, all of these rules work together
to shore up the central design theory of "modularity," which
stipulates that components are self-contained and interoperable, so
they can be independently created, modified, and replaced without
affecting the whole system.

In 21st century computing, hegemonic social ideologies have continued
to spread into data gathering, surveilliance, and quantification
practices. As computational power grows, the emphasis on efficiency
perpetuates social stratifications from previous eras, a situation
that Ruha Benjamin calls "The New Jim Code." As Benjamin asserts, "the
road to inequity is paved with technical fixes," where newer
technologies reproduces social inequities under the guise of
objectivity and progressivism (7). While newer technology is
continually promoted as more and more efficient, it masks the ways
that it gathers data about its subjects. Turning to Artificial
Intelligence and algorithmic computing, Benjamin explores how
innovations in tracking, labelling, and monetizing data extend racist
paradigms into ever new tools, for example, in databases for financial
services that associate "black names" with criminality (Benjamin
5).\footnote{Alondra Nelson's foundational work explores how blackness is
constructed in media, which is anti-technology, anti-progress, while
the same media is predisposed to present whiteness as invisible,
universal, disembodied, an quality that does not hold weight, is not
marked. See Nelson, Alondra. “Future Texts.” \emph{Social Text} 71, Vol. 20,
No. 2, Summer 2002. For more on blackness and tracking technology, see
Browne, Simone. \emph{Dark Matters: On the Surveillance of Blackness}. Duke
University Press, 2015.} Benjamin explains, "we are told that how tech sees
"difference" is a more objective reflection of reality than if a mere
human produced the same results\ldots{} bias enters through the backdoor of
design optimization in which the humans who create the algorithms are
hidden from view" (5-6). Like the creators of UNIX, the creators of
such tools and algorithms operate under assumptions that mark
difference as "other."

The second assumption is about data, and particularly, "raw data."
Johanna Drucker, who examines data visualization practices, is careful
to dispell the illusion that data is initially gathered or processed
in an unaltered state. Because data always undergoes a transformation
into electronic format, Drucker explains, its complexity is always
already compromised. As a result, quantification techniques such as
visualizations in graphs and charts inescapably represent data that
has already been transformed. Each piece of data carries with it the
result of many interpretive decisions, that carry with them varying
degrees of opacity, which are all necessary in order to present
complex concepts. Drucker explains: "the graphical presentation of
supposedly self-evident information\ldots{} conceals these complexities,
and the interpretative factors that bring the numerics into being,
under a guise of graphical legibility" (Drucker par. 23). To resist
the reductions of data, a term that deceptively connotes that which is
"given," Drucker proposes thinking of data as "capta," which suggests
that which is taken. Drucker's "capta" is deliberately creative,
turning graphical expressions into expressive metrics: components used
for measurement, like lines or bars on a graph, break, blur, or bleed
into one another. Objects are not discrete entities, but interact with
the other objects in the visualization. Emphasizing "capta" is a way
of figuring elements that have been reduced, resolved, or ignored in
traditional quantitative analysis.

\subsection{incommensurability}
\label{sec:org84c739e}
Unlike technological processes that seek to transform and manage
information about real-world objects into computable data, queer
methods of analysis often seek to surface that which eludes capture or
categorization. Building from Queer Studies scholars José Esteban
Muñoz and Gloria Anzaldúa, this dissertation defines queerness as an
incommensurable quality characterized by sensations of estrangement,
displacement, and even physical upheaval. According to Muñoz, queer
subjectivity grows from an affective experience of "disidentification"
in which minority subjects negotiate identity within majority culture
(\emph{Disidentifications} 5).\footnote{Muñoz builds from Chicana theorists Norma Alarcón's idea of
"differential consciousness" and Chela Sandova's concept of emergent
identities-in-difference, which center moments of failed
interpellation as the core materials of subject formation, to a
general paradigm of identity formation that he calls
"identities-in-difference" (\emph{Disidentifications} 6).} Here, minority experience is defined
by a \emph{gap} in identification, where subjectivity emerges in the
failure to adhere to social expectations. Within this gap, minority
subjects find alternative pathways to connect with dominant
signfications of identity, "read[ing] onesself and one's own life
narrative in a moment, object, or subject that is not culturally coded
to 'connect' with the disidentifying subject" (\emph{Disidentifications}
12). One way that disidentification manifests is physical, evoking to
what Chicana theorist Gloria Anzaldúa describes as \emph{el choque}, a
bodily experience of collision betwen two opposing forces. In the
experience of the choque, the subject receives opposing cultural
messages that incite a physical upheaval. Anzaldúa explains that this
experience occurs in those like the \emph{mestiza}, who cross lines of
gender, race, language, sexuality: "Cradled in one culture, sandwiched
between two cultures, straddling all three cultures and their value
systems, \emph{la mestiza} undergoes a struggle of flesh, a struggle of
borders, an inner war" (Anzaldúa 78).

The experience of disidentification, and its physical (and sometimes
painful) manifestation in the choque, enables incommensurable elements
of queerness to surface.  I take this term "incommensurable" from
Latina feminist philosopher Ofelia Schutte, who defines it as "a
residue of meaning that will not be reached in cross-cultural
endeavors" (56). Drawing from feminist postcolonial and
poststructuralist concepts of alterity and difference, Schutte
theorizes ambiguity as politically potent tool for cross-cultural
communication. Schutte gives an example of how the incommensurable
emerges in conversation:
\begin{quote}
In cross-cultural communication, each speaker may "say" something that
falls on the side of the "unsaid" for a culturally differentiated
interlocutor. Such gaps in communication may cause one speaker's
discourse to appear incoherent or insufficiently organized. To the
culturally dominant speaker, the subaltern speaker's discourse may
appear to be a string of fragmented observations rather than a unified
whole. The actual problem may not be incoherence but the lack of
cultural translatability of the signifiers for coherence from one set
of cultural presuppositions to the other. 62
\end{quote}
Schutte proposes that one embrace the strangeness of communication,
locating the moments where meaning seems to slip by and elude us. The
point of isolating incommensurability is not to try to grasp or
translate the vestige of lost meaning, but to recognize that gap as a
space that constitutes queer experience and subjectivity. As a moment
of failure, where meaning does not transfer, incommensurability
describes a productive effect of the embodied experience of
disidentificaiton. Attending to such gaps and ellisions, to the ways
in which, for example, "the other's speech, or some aspect of it,
resonates\ldots{} as a kind of strangeness, as a kind of displacement of
the usual expectation," offers productive material for understanding
queer (dis)identity (56).

Moments of strangeness, dissociation, and even failure or loss enable
queerness to retain a quality of elusiveness. This quality has been
usefully deployed in the subfield of Queer Historiography, where it
drives a critical methodology that describes or acknowledges, without
attempting to capture or redeem, the failure and loss that defines the
experience of queer subjects throughout history. Resisting the urge to
negative histories into sites of resistance or affirmation, Heather
Love, for example, surfaces moments where queer subjects turn away
from intelligibility. Love points to the Greek myth of Orpheus and
Eurydice, to the moment when Orpheus turns to gaze at his lover as he
attempts to save her from the Underworld, a moment in which Orpheus
deliberately botches the rescue. Love quotes from Maurice Blanchot's
account of the story in "The Gaze of Orpheus," where Blanchot describe
why Orpheus cannot resist looking back and thus dooming his lover:
\begin{quote}
Not to look would be infidelity to the measureless, imprudent force
of his movement, which does not want Eurydice in her daytime truth and
in her everyday appeal, but wants her in her nocturnal obscurity, in
her distance, with her closed body and sealed face---wants to see her
not when she is visible, but when she is invisible, and not as the
intimacy of familiar life, but as the foreignness of what excludes all
intimacy, and wants, not to make her live, but to have living in her
the plenditude of death. 50
\end{quote}
Orpheus's downfall is his desire for a glimpse at what cannot be
grapsed. This desire is for his love in the image of "noctural
obscurity," an image of her perpetually receding into the
Underworld. Love argues that like Eurydice, the image of queerness
depicts that which is eternally slipping away, turning its face from
the gaze of the critic. Rescuing Eurydice, wrenching her from her
suffering, would effectively transform her into something
else. Rather, as Love argues, this kind of desire seeks what which
cannot be grasped or transformed.

\subsection{queer forms}
\label{sec:org532c65d}
This disseration proposes an analytical method that apprehends
queerness in an abstracted form, but a form that is not any less
sensual for its abstraction. Here, I pose the incommensurable
qualities of queer identity and experience against the necessary
disambiguiation of technological processes. First, through digital
methods like text analysis, text encoding, and media archaeology
(discussed in more detail below), I seek out elements of queerness
that resist the transformation between technical registers, elements
that are constituted through lack or displacement. Then, I explore how
digital tools might be reworked to engage with this resistance. To
explore forms that elude representation, I combine the concepts of
deformance (from Digital Studies) and queer form (from Queer
Studies). Deformance, coined by Jerome McGann and Lisa Samuels,
describes the act of distorting, disordering, or re-assembling
literary material, estranging the reader from their familiarity of the
text, as a critical method for revealing interpretive
possibilities. According to McGann and Samuels, although electronic
formats reduces complex literary elements into to computable
components, it also confronts the reader with new opportunities for
analysis. The key here is form: by continually subscribing the text to
new configurations, digital tools expose semantic potentialities of
the text's latent aspects. The process of deformance reworks literary
text into "queer forms," a term I borrow from Kadji Amin, Amber
Jamilla Musser, and Roy Pérez to describe "an aesthetics that moves
persistently around the visual," "mak[ing] difference a little less
knowable, visible, digestible" (235). Queer form, according to these
theorists, "queer formal practices can resist the dictates of
transparency normally required of non-normative subjects by
illuminating the unseen" (233). Queer form effectively figures the
sensuality of the contour, boundary, and edge that outlines the
elusive identities, repressed desires, and other coded elements of
queerness in text. To seek out this queer form, I examine material
specificities of eletronic formats and how they enable numerable
interventions upon the textual object.

My dissertation includes a digital component that demonstrates in
practice how these tools reveal, not solutions for understanding or
"fixing" queerness, but opportunities for exploring its ever shifting
permutations. As a practical application of my research, this digital
component, called the Queer Text Toolkit," offers an introductory way
of exploring the interpretive possibilities of queer form in text
analysis and text encoding procedures. Here, users can experiment
firsthand with how reductive digital formats and processes, which
collapse stylistic and formal expressions of gender, sex, and
sexuality into computable data, can be redeployed toward creative and
radical exploration. Aimed at an audience of humanist scholars at the
beginning of their technical training, the toolkit offers a blueprint
that lowers the barrier to entry for educators and students using
digital tools to work with queer literature.

The project consists of two applications, "queer distant reading" and
"queer text encoding," which build from open source software and
standards for quantitative text analysis and semantic markup
procedures. The "queer distant reading" application is a command-line
application that walks users through text analysis procedures inspired
by Judith Butler’s theory of gender performativity. Here, users
experience firsthand how the process of iterating over text, which is
central to text analysis tasks, draws from Butler’s formulation of
gender as a series of repeated acts that destabilize binary structures
of gender. The application consists of a Python module containing
scripts for loading, cleaning, analyzing, and visualizing the text, as
well as a small test suite, which builds from open source Python
libraries like the Natural Language Toolkit and NetworkX. The "queer
text encoding" tool offers an interactive and beginner-friendly Text
Encoding Initiative (TEI) workflow for "marking up" homoerotic content
in text. The website interface encourages readers to think
productively about the limitations of discrete labeling protocols and
how this work engages with Queer of Color Critique on destabilizing
hegemonic power structures. The tools consists of a JavaScript-based
web application containing a transcribed and encoded manuscript of a
portion of Oscar Wilde’s \emph{The Picture of Dorian Gray}, which Wilde
edited to remove suggestions of homoeroticism.

\subsection{chapter trajectory}
\label{sec:org81e515e}
Besides offering new digital procedures for studying textual material,
my work also poses a crucial critique of Queer Studies and the way it
theorizes the relationship between sex, gender, sexuality, and
race. The order of chapters in my dissertation poses a trajectory for
the field of Queer Studies that increasingly grapples with the role of
race in queer identity. My first chapter on text analysis considers
early formulations of queerness as a discursive phenomenon,
exemplified by Judith Butler’s theory of gender performativity, which
was heavily critiqued for eliding the lived realities of queer
embodiment. Then, my second chapter on text encoding traces how, in
the wake of mainstream acceptance, Queer of Color Critique clears a
way for resisting the pressures of heteronormative and neoliberal
politics. My last chapter, energized by Black Feminist and Chicana
Feminist thinking that powers much of Queer of Color Critique, embarks
on a close reading of the sensual and material aspects electronic
media that offers possibilites for new modes of social relation.

My first chapter, "'A Melon, an Emerald, a Fox in the Snow':
Quantifying Gender in Virginia Woolf's \emph{Orlando: A Biography},"
examines how computational text analysis grapples with gender ontology
in Woolf's novel, \emph{Orlando}, which features a transgender
protagonist. The chapter begins by tracing how the adoption of
quantitative methods to analyze gender in Literary Studies perpetuates
assumptions about gender as binary. I contrast this "reproducible"
approach with more experimental ones that use quantitative methods to
deconstruct social categories like gender and race. Then, the middle
portion of the chapter draws connections between computer programming
and gender theory. First, it delves into python programming, focusing
on the principle of iteration that drives cleaning and regularizing
tasks, as well as the transformation of words into numerical
representations for quantitative processing, with the goal of bringing
out the iterative quality of working with python code. It then moves
to Judith Butler’s concept of gender performativity, which posits how
gender expression might subvert traditional social structures through
repeatedly "performing" gender constraints in ways that deviate from
the norm. Taking this shared quality of iteration between python and
gender, I propose a text analysis methodology that interweaves, or
iterates through, distant and close reading. Turning to Woolf’s text,
I then demonstrate how this method of text analysis leads to a
plurality of significations for gender terms in the novel, revealing
how language and gender are closely coordinated in the narrative. I
conclude by considering the limitations of this method, which poses
gender as a discursive phenomenon, and its place within a larger
trajectory of Gender Studies since Butler's text inaugurating the
field.

My second chapter, "'Where there is Spectacular Passion, they would
Suggest Something Vile': Encoding Queer Erasure in Oscar Wilde’s \emph{The
Picture of Dorian Gray}" consists of two parts: the first explores the
Text Encoding Initiative (TEI) standard, an electronic editing tool
that allows researchers to "mark up," or tag, textual elements, to
encode the homoerotic elements that Wilde edited during his revisions
of Dorian Gray (1890). My analysis in this section finds that the TEI
works best with data which is discrete and bounded, rather than smooth
data. Like my critique of text analysis, this constraint reveals a
connection to queerness: As a labeling tool, the TEI surfaces moments
where queer themes, which are plural and permeable in this text,
threaten to spill over the bounds of its data structure. I close this
first section by proposing a custom editorial workflow that encourages
editors to tag the homoerotic elements in such a way that allows them
to retain some of their elusiveness. Then, in my second section, I
delve deeper into the mutually reinforcing nature of dominance
structures across data formats and text encoding practices. Here, I
draw from Queer of Color's Critique on Queer Studies and Black
Feminist scholarship on the archive of slavery to energize a radical
re-thinking of editorial practices. I close by highlighting examples
of current projects that deploy collaborative and minimalist practices
to challenge the structuring modes of textual editing and the TEI.

Whereas the first two chapters are about deconstructing digital tools,
specifically for text analysis and text encoding, my third chapter,
"Sex, Flesh, Skin: A Media Archaeological Reading of \emph{Dawn} and
\emph{skinonskinonskin}" engages a close reading of electronic media. This
chapter juxtaposes two unlikely texts—-an early hypertext work from
1999 (\emph{skinonskinonskin}, written by net artist Entropy8Zuper!), and a
science fiction novel from 1987 (\emph{Dawn} by Octavia Butler)—-to unpack
the role of media and mediation across physiological and technological
systems. Though these works present vastly different narrative worlds,
not to mention physical formats, they both trouble the boundary
between materiality and abstraction, in one case technologically,
through computer hardware and software, and in another
physiologically, through nervous systems and brain chemistry. Here, I
read for sensuality across these medial environments in each text. In
my analysis, this concept of materiality, expressed by hardware and
human flesh, becomes a ground for understanding how physical registers
interact with symbolic ones. Drawing from thinkers in Chicanx Studies
and Black Feminist Studies, I take the systematic reduction of the
Black body to the physical flesh, a process that began during the
violences and atrocities of the Middle Passage, as an opportunity for
rethinking the political potential of pleasure and its relationship to
racialization in each text.

This critique of Queer Studies and its implicit whiteness
strategically poses queerness as something that eludes definition or
confinement. For queerness, as Muñoz argues, is "not yet here," but
perpetually on the horizon (1). At the end of this project, queerness
remains a target beyond reach, a fount for future subversions,
exemplified with the term "queer" itself, which Butler famously says
is "never fully owned, but always and only redeployed, twisted,
queered from a prior usage and in the direction of urgent and
expanding political purposes" (173). Emphasizing the nebulous and
shifting nature of this term, this work offers an approach for
studying queer texts that does not fully circumscribe to a general
methodology. The goal is not to build reproducible schemas and models
for analyzing queerness. Rather, it is to harness opacity and
unintelligibility as resources for resisting inclusion into what Muñoz
describes as "the ossifying effects of neoliberal ideology" (22). My
project, as I try to demonstrate with the "Queer Text Toolkit"
application, posits queer form as a kind of technology of resistance,
which digital tools can help to surface. I hope this experimental work
will encourage the further developments in reading our queer literary
heritage, which, as Butler says, "begin, without ending, without
mastering, to own—and yet never fully to own—the exclusions by which
we proceed" (25).

\subsubsection{sensual fullness of a lack}
\label{sec:orgb6ea6f4}
\begin{itemize}
\item Returning to the question about \emph{Waves}: How does its structure
relate to queerness?
\item A lack of fulfillment/total understanding draws on this idea that
queerness strategically lacks a teleology.
\begin{itemize}
\item Insert: \emph{Confessions} reading?
\item Munoz: queerness is not here. Queerness is a structuring mode of
desiring.
\item What does this gain us? A way of reading touch that is more
concerned with exploring the potential of a touch rather than
verifiable contact. The sensual fullness of a lack.
\end{itemize}
\end{itemize}

TEXT:

The reader’s confusion in navigating through \emph{Waves}, in
re-interpreting fragments that had been previously integrated,
reinforces desire, and queer desire in particular, as something
elusive, a condition that is not fully intelligible. Clicking
(touching) her way through the narrative, the reader is repeatedly
reminded of her removal from the story, in her inability to grasp the
story. This work thus literalizes the connection between touch, its
frustration, and queer desire.

It is interesting that this medium makes the story accessible through
touch (the “click” on the hyperlinks) while not giving full visual
access to the individual narratives like a traditional print work
(which is a phenomenon exaggerated in the ticker tape). 

\section{commands}
\label{sec:org2a37f86}
c-c c-x f => create a new footnote
c-u c-c c-x f then select s => renumber footnotes

block quotes: \#+BEGIN\(_{\text{QUOTE}}\) \& \#+END\(_{\text{QUOTE}}\)

\section{works}
\label{sec:org5e08907}

Amin, Kadji, Amber Jamilla Musser, and Roy Pérez “Queer Form:
Aesthetics, Race, and the Violences of the Social” ASAP/Journal,
Volume 2, Number 2, May 2017, pp. 227-239.

Barnett, Fiona, Zach Blas, micha cárdenas, Jacob Gaboury, Jessica
Marie Johnson, and Margaret Rhee. “QueerOS: A User’s Manual.” \emph{Debates
in the Digital Humanities}. 2016.

Benjamin, Ruha. \emph{Race After Technology: Abolitionist Tools for the New
Jim Code}. Polity, 2019.

Blas, Zach and micha cárdenas. Queer Technologies / TransCoder.
2007-2012.  Butler, Judith. Bodies That Matter: on the Discursive
Limits of Sex. Routledge. 1993.

Browne, Simone. \emph{Dark Matters: On the Surveillance of Blackness}. Duke
University Press, 2015.

Calado, Filipa. 2022. “Encoding Queer Erasure in Oscar Wilde’s The
Picture of Dorian Gray”, Open Library of Humanities 8(1). doi:
\url{https://doi.org/10.16995/olh.6407q}

Drucker, Johanna. “Humanities Approaches to Graphical Display.” DHQ:
Digital Humanities Quarterly. Vol 5, No 1. 2011.

Entropy8Zuper!
skinonskinonskin. Rhizome. \url{https://anthology.rhizome.org/skinonskinonskin}

Gaboury, Jacob. “Becoming Null: Queer Relations in the Excluded
Middle.” Women \& Performance: A Journal of Feminist Theory, vol. 28,
no. 2, 2018, pp. 143–158.,
\url{https://doi.org/10.1080/0740770X.2018.1473986}.

Hartman, Saidiya. "Venus in Two Acts." \emph{Small Axe}, vol. 12 no. 2,
2008, p. 1-14.

Hayles, Katherine. How We Became Posthuman: Virtual Bodies in
Cybernetics, Literature, and Informatics, 2000.

Johnson, Jessica Marie. Wicked Flesh: Black women, Intimacy, and
Freedom in the Atlantic World. University of Pennsylvania Press, 2020.

Keeling, Kara. "Queer OS." Cinema Journal, vol. 53 no. 2, 2014,
p. 152-157. Project MUSE, \url{doi:10.1353/cj.2014.0004}.

Klein, Lauren F. “The Image of Absence: Archival Silence, Data
Visualization, and James Hemings.” American Literature. 85
(4), 2013. pp. 661–688.

Kirschenbaum, Matthew. Mechanisms: New Media and the Forensic
Imagination. 2008.

Love, Heather. Feeling Backward: Loss and the Politics of Queer
History. 2009.

Mandell, Laura. “Gender and Cultural Analytics: Finding or Making
Stereotypes?” Debates in Digital Humanities 2019. Ed. Matthew K. Gold
and Lauren Klein. University of Minnesota Press, 2019.

McPherson, Tara, “U.S. Operating Systems at Mid-Century: The
Intertwining of Race and UNIX.” In Race after the Internet, ed. Lisa
Nakamura and Peter A. Chow-White, 21–37. New York: Routledge, 2012.

Moretti, Franco. Graphs, Maps, Trees: Abstract Models for Literary
History. 2007.

McGann, Jerome, and Lisa Samuels. “Deformance and Interpretation,”
Radiant Textuality: Literature after the World Wide Web. 2001.

Muñoz, José Esteban. Cruising Utopia: The Then and There of Queer
Futurity. NYU Press. 2009.

Muñoz José Esteban. \emph{Disidentifications: Queers of Color and the
Performance of Politics}. University of Minnesota Press, 1999.

Musser, Amber Jamilla. Sensual Excess: Queer Femininity and Brown
Jouissance. NYU Press, 2018.

Nelson, Alondra. "Future Texts." \emph{Social Text} 71, Vol. 20, No. 2,
Summer 2002.

Pope, James. "The Significance of Navigation and Interactivity Design
for Readers' Responses to Interactive Narrative: Some Conclusions from
an Empirical Study of Readers' Responses." \emph{Dichtung Digital. Journal
für Kunst und Kultur digitaler Medien}, No. 39. 2009. pp. 1-22.

Ramsay, Stephen. Reading Machines: Toward an Algorithmic
Criticism. 2011.

Ruberg, Bonnie et al. “Toward a Queer Digital Humanities.” Bodies of
Information, edited by Elizabeth Losh and Jacqueline Wernimont,
University of Minnesota Press, 2018, pp. 108–28.

Snorton, C. Riley. Black on Both Sides: A Racial History of Trans
Identity. University of Minnesota Press, 2017.

So, Richard Jean and Edwin Roland. “Race and Distant Reading, PMLA
Special Topic: Varieties of Digital Humanities. Vol. 35,
No. 1. January 2020. pp. 59–73.  Spillers, Hortense J. “Mama’s Baby,
Papa’s Maybe: An American Grammar Book.” Diacritics, vol. 17, no. 2,
1987, pp. 65–81.

Wilde, Oscar. The Picture of Dorian Gray. 1890, 1891 \& Manuscript.
Woolf, Virginia. Orlando: A Biography. Hogarth Press. 1928.
\end{document}
