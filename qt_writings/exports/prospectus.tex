% Created 2019-09-07 Sat 20:03
% Intended LaTeX compiler: pdflatex
\documentclass[11pt]{article}
\usepackage[utf8]{inputenc}
\usepackage[T1]{fontenc}
\usepackage{graphicx}
\usepackage{grffile}
\usepackage{longtable}
\usepackage{wrapfig}
\usepackage{rotating}
\usepackage[normalem]{ulem}
\usepackage{amsmath}
\usepackage{textcomp}
\usepackage{amssymb}
\usepackage{capt-of}
\usepackage{hyperref}
\author{Filipa  Calado}
\date{\today}
\title{}
\hypersetup{
 pdfauthor={Filipa  Calado},
 pdftitle={},
 pdfkeywords={},
 pdfsubject={},
 pdfcreator={Emacs 26.2 (Org mode 9.1.9)}, 
 pdflang={English}}
\begin{document}

\tableofcontents

\section{Prospectus\hfill{}\textsc{diss}}
\label{sec:org46ebdbe}

\#+ Queer Tools

\subsection{Problem Statement: Utility of DH Tools}
\label{sec:org3920e27}

There has been much debate about what digital tools can do for the
humanities, and literary studies in particular.\footnote{One way to understand the variety of ways that DH practitioners
approach their work is to place them on a spectrum, based on their
faith in the ability of methods such as computation, quantification,
visualization, and encoding to pursue and answer questions about
literature. These theorists take different stances on whether DH tools
facilitate a more objective, or what Franco Moretti calls
“falsifiable," method of criticism, where graphs and other
visualizations might answer questions about literary history and form,
or a more “speculative” mode, championed by Johanna Drucker, who
purposefully skews graphical metrics in order to reflect the ambiguity
and partiality of the data they represent. In between these two
extremes, there are theorists like Andrew Piper and Ted Underwood, who
temper the reductions and rhetoric of "distant reading" by
incorporating close readings and explanations, or Catherine D'Ignazio
and Lauren Klein, whose attempts to recover emotion and emobodiment
are a direct indictment on data science as an exclusionary discipline.} Although many DH
practitioners have been careful to deliniate the purpose of such tools
within the context of humanistic study, there persists an idea that
digital methods ought to verify, correct, or establish facts about
literature and literary history, as it does in the sciences. Stephen
Ramsay refutes this position, pointing out that, while scientific
inquiry aims to describe the 'real world,' the humanities, approach
observation as a phenomenal experience. Drawing from Lisa Samuels' and
Jerome McGann's concept of "deformance," Ramsay takes constraints of
computation as an opportunity for reflection on the ways that critics
change and transform the texts under their scrutiny. He reminds us
that these tools function within a larger rhetorical process: “The
understanding promised us by the critical act arises not from a
presentation of facts, but from the elaboration of a gestalt, and it
rightfully includes the vague reference, the conjectured similitude,
ironic twist, and the dramatic turn" (16). The issue, though, is a
recurrent one. More recently, Nan Z Da asserts the incompatability
between Computational Literary Studies (CLS) and humanistic
inquiry. Da deems CLS ineffective when its methods fail to reproduce
the results of her colleagues. Her emphasis on the "reproducible"
extends Franco Moretti's call---all the way from 1983---for a
"falsifiable criticism." Indeed, though they have different endgoals,
Moretti, Da and Ramsay all enlist the heightened objectivity made
possible by the machine to do subjective work.

\subsection{Proposal}
\label{sec:orgaa4f168}
This project interrogates this assumption that the results of using
digital tools ought to be reproducible. Instead, it prosposes that we
mess around with tools in ways that resist the impulses for fixity,
correction, certainty, verification. Such a method values
experimentation exploration and revels in the inconsistencies, errors,
and idiosyncracies of close reading. It examines the ways that both
human minds and computers are prone to making mistakes, and takes
these ironies as points of inspiration for human-computer
interactions. This project looks specifically at digital methods that
engage reading and interpretive processes, such as electronic editing
practices and social reading tools. Here, I’m interested in using the
“digital” to mine and/or imagine terrains of affect and cognition in
queer texts. This examination draws me into studies of consciousness,
where I look at cognitive and embodied perceptual processes and what
they might reveal about the reading experience. My goal with this
project is to explore how digital methods and media can be used to
engage queer subjectivity, embodiment and affect. How do the logics
and limitations of technology (struggle to) reflect the complexities
or queer subjectivity or gender identity? Is it possible to encode or
decode queerness? I explore a variety of parallels between human and
computer as starting points to speculate on how the affinities between
man and machine might emerge in the interface between reader and text.

\subsection{Stake: Emoting Machines:}
\label{sec:org2191072}
This project aims to intervene on what digital humanities has to gain
from Queer Theory. At stake here is the critic's relation to the
object of study. Some Digital Humanities practitioners have already
offered useful models for reading that emphasize the critic's role in
analysis, such as Ramsay's "algorithmic criticism," Jerome McGann's
"deformative criticism," and Johanna Drucker's "Speculative
Computing." These approaches are careful to temper the common
reductions and rhetoric of "distant reading" by being explicit about
the role of critical self-awareness in their analysis. Similarly,
Queer Theory has been exploring new ways of relating to textual
objects in ways that do not delimit them. But, unlike DH, it has also
been forced to contend with affect. Queer Theory, particularly in the
past several years, is plagued by what to do with bad feelings that
arise when confronting histories of queer exclusion, repression, and
violence. The issue, according to Eve Sedgwick, is that critics tend
to pursue "the heroic, ‘liberatory,’ inescapably dualistic
righteousness of hunting down and attacking prohibition/repression in
all its chameleonic guises” (10). To avoid perpetuating the logic of
exposure that guides much of "suspicious" or "paranoid" reading,
theorists like Sedgwick offer models of relationality harness the
critical possibilities of "touch" as a way of making connections
without delimiting, recovering, or redeeming the object of
study.\footnote{Though Sedgwick focuses on reconceptualizing
traditionally negative affects (like shame) into creative resources,
and Love prefers to dwell in the painful past, giving inconsolable
characters full reign over their own darkness, both critics look at
touch as a new way of making connections.} Throughout my project, Sedgwick's idea of
"touching/feeling," looking "beside rather than beneath," joins forces
with Heather Love's idea of "touching withought touching" to offer a
model of movement that resists the temptations to verify, establish,
or fix knowledge about queer subjects.

\subsection{Cyborgs}
\label{sec:org84d68ec}

Myths about computation that do not hold on the immaterial level. 

Kirschenbaum's screen essentialism

“Commonplace assumptions about …. Ephemerality, fungibility, and
homogeneity. A forensic perspective furnishes us with two key concepts
for an alternative approach to electronic textual studies: trace
evidence and individualization. Ultimately, electronic data assumes
visible and material form through processes of instrumentation that
suggest phenomena we call virtual are in fact physical phenomena
lacking the appropriate mediation to supplement wave-length optics”
(19).

\subsection{Life Writing by Queer Authors}
\label{sec:org45316f5}
\subsubsection{Why life writing?}
\label{sec:org17d68c3}

This investigation takes "life writing", which includes autobiography,
memoir, journals, manuscripts, as well as fictionalized autobiography,
biography and roman a clef, as its object of study. I use the term
"life writing" loosely: the body of texts in my study do not adhere to
any conventional criteria of genre, except that all of them either
demonstrate a writing self or an experimental style that reflects the
fragmentary and processural nature of subjecthood. My interest in life
writing emerges from a concern with the ways that literary form shapes
the coherence of queer identity and engages embodied affects in
subjective experience. The focus on composition, particularly in the
manuscripts and journals of Oscar Wilde and Virginia Woolf, explores
how writing is an iterative practice that continually calls into
question the delimitations of embodiment and subjectivity. Readings
from what I call 21st century "memoiresque," including works by Alison
Bechdel, Jordy Rosenberg, and Yiyun Li, function as present day
examples of life writing that continues to pose fresh problems for
understanding queer identity. A recurring theme in my analysis is the
difficulty of categorizing and decoding or encoding queer identity,
especially when affective responses to reading get in the way of
interpreation. I am also wary of overanalyzing and delimiting queer
subjecthood. Is it possible to mark queerness as legible so that it
remains dynamic? How do postmoderns writers approach the differences
bewteen fiction and memoir? Woolf's ficitonal biography, \emph{Orlando},
acts as a fulcrum around which the false distinctions of fiction and
nonfiction, writing and the self, revolve. I wonder if these generic
boundaries are even useful.

\subsection{First Chapter: 'Touching but not Touching': Toward a Queer DH}
\label{sec:org624a65d}

The first chapter, "Toward a Queer DH," proposes what Digital
Humanities methodologies might learn from queer theory. It looks at
specific reading practices by Eve Sedgwick and Heather Love, which
question the proper relationship of the critic to the object of
study. Sedgwick's ideas about "touching/feeling" and Love's notion of
"touching but not touching" pose queer modes of relationality that do
not presume full connections as prerequisite for analysis. This
chapter engages these techniques with various critical methods, many
from the Digital Humanities, such as distant, surface, and
post-critical reading, and deformative, algorithmic, and speculative
criticism. I pay particular attention to how digital formats and
interfaces might harness queer critical methods, speculating on the
intimate process of engaging with texts on a computer, where users can
manipulate and transform texts in virtually infinite and unique
ways. Examining two digital projects, a text analysis and
visualization web application called \emph{Voyant-Tools}, and a hypertext
novella by Caitlin Fischer, \emph{These Waves of Girls}, I reflect how the
tools work alongside the reader’s intuition, in what Jerome McGann
calls a “prosthetic extension of that demand for critical reflection,”
by which the reader is able to feel her way through the text. In
particular, I examine the haptic and exploratory methods might enliven
the reading process by allowing the reader to play, experiment, and
imagine new connections to the textual object.

\subsection{Second Chapter: Editing: Encoding Identity}
\label{sec:org6449bfa}

The second chapter poses two contrasting ideas---the instability of
queer identity and subjecthood against the limitations and fixities of
computational methods. How do the abstractions and structures of
digital formats struggle or fail to mark the complexities of embodied
experience? To answer this question, I first review the relation
between queerness and loss, failure and negative affects. Then, I
explore how TEI (the Text Encoding Initiative), an encoding standard
or "markup language" for electronic editig, struggles to portray the
complexity of the data it represents. Here, I demonstrate how TEI
might be used to mark up the manuscript of Oscar Wilde's \emph{The Picture
of Dorian Gray}, which was twice edited by Wilde in ways that
minimized homoerotic content, among other changes. I question how TEI
as a standard enables editors to mark up such revisions, given the
hierarchical structure of XML (TEI's parent language) and the
available relevant "tags" for marking up manuscripts. I bring my
investigation into conversation with other critics and teachers who
have found difficulty with their own deployments of TEI: Pamela
Caughie, who's attempts to encode gender in Lili Elbe's memoir, \emph{Man
Into Woman}, turn a question of about technical difficulty back to the
difficulty of pinning down sexual identity: "Can ontologies ever
capture the complex, multi-layered, dynamic nature of gender
identities?” Throughout my conversation about \emph{Dorian Gray}, I bring
up examples of 21st century queer memoirs who have experimented with
alternative ways of figuring queer identity and subjecthood. This
chapter ends by delving into the complex question of gender ontology
and sexual identity, which are never fully legible despite our tools.

\subsection{Third Chapter: Teaching: Social Annotation}
\label{sec:org93c066f}

The third chapter questions how digital tools might facilitate
embodied interactions with text. To begin to answer this question, I
examine biological processes of perception and analogies between man
and machine. I take the figure of cyborg to explore
biological-mechanical intersections, such as neurons and transistors,
memory and RAM, in order to re-present the human as a creative
machine, a thinking/feeling machine. This human-computer engagement,
rather than reduce the differences between humans and machines, allows
me to explore unexpected affinities that emerge in the interface
between reader and text. Understanding how certain perceptual and
computational processes work across biological and mechanical levels,
I argue, will spark our thinking about how humans might interact with
machines. I take up social annotation as a test case, examining a
particular digital annotation tool that I've modified for purposes of
engaging preverbal and nonverbal embodied and affective responses
while reading. This examination ends with a close reading of Virginia
Woolf and Djuna Barnes novels, to see both how their prose evokes some
of the perceptual processes involved in reading and writing as well as
how digital annotation practice helps readers to recognize these
processes. How might formal strategies developed by modernist writers
engage cognitive/affective/embodied processes?

\subsection{Fourth Chapter: Displacement: Feeling Backward / Feed Forward}
\label{sec:orgb751409}

The fourth chapter, "Queer Temporalities: Feeling Backward / Feed
Forward," explores how displacements across our understanding of
media, perception, and language creates opportunities for critical
speculation. Here, various displacements between what we
see/experience and how it works/functions converge to offer a model
for thinking about how we interact with digitized literary and
cultural objects. In digital media, there is a central displacement
between what we see on the screen and what is inscribed in the
computer’s hardware---what Matt Kirschenbaum refers to as the levels
of formal and forensic materiality. Within the neuroscientific and
philosophical debates about consciousness, the central displacement is
the “explanatory gap," or the idea that the physiological processes of
the body, such as the neurons firing in the brain, cannot explain the
quality of sensation that is experienced by each individual. With
language, displacement inheres in the instability of the signifier,
and the way that ideas are passed through words, syntactic structures
and sound. This chapter examines how digitization practices of textual
objects engage these displacements to open up our experience and
interpretation of them. I look at the examples of the Lesbian Herstory
Archive and "Comparing Marks: A Versioning Edition of Virginia Woolf's
'A Mark on the Wall.'" I find in the displacements an occasion for
thinking through how emergent perceptual experiences (Mark Hansen's
idea of "Feed Forward") might contend with backward looking or
negative affects of queer theory (Love's model of "Feeling Backward").

\section{Works Cited}
\label{sec:orgb967482}

Moretti, \textbf{signs} 
Ramsay, \textbf{algorithmic} 
Drucker, DHQ article 
CatherineD'Ignazio and Lauren Klein \textbf{Data Feminism} 
Andrew Piper, \textbf{Enumerations} 
Ted Underwood, \textbf{Distant Horizons} 
Nan Z Da, Critical Inquiry article 
Lisa Samuels, Jerome McGann, "Deformance\ldots{}"  
Micki Kaufmann, *Quantifying Kissinger"
Sedgwick
Love
McGann, Radiant Textuality
Hansen, Feed Forward
Tougaw, Elusive Brain
Lesbian Herstory Archives
Versioning Marks
\end{document}
