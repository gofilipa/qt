% Created 2022-10-21 Fri 13:01
% Intended LaTeX compiler: pdflatex
\documentclass[11pt]{article}
\usepackage[utf8]{inputenc}
\usepackage[T1]{fontenc}
\usepackage{graphicx}
\usepackage{grffile}
\usepackage{longtable}
\usepackage{wrapfig}
\usepackage{rotating}
\usepackage[normalem]{ulem}
\usepackage{amsmath}
\usepackage{textcomp}
\usepackage{amssymb}
\usepackage{capt-of}
\usepackage{hyperref}
\author{Filipa  Calado}
\date{\today}
\title{}
\hypersetup{
 pdfauthor={Filipa  Calado},
 pdftitle={},
 pdfkeywords={},
 pdfsubject={},
 pdfcreator={Emacs 26.2 (Org mode 9.1.9)}, 
 pdflang={English}}
\begin{document}

\tableofcontents

\section{Containment and Queerness}
\label{sec:org80903e0}
\subsection{requirements}
\label{sec:org0a0ad21}
\textasciitilde{}3-6K words total, 12 - 24 pages
chicago, 2 images max, docx format
\subsection{research}
\label{sec:orgee478a5}
\subsubsection{QOCC}
\label{sec:org4a867ef}
\begin{enumerate}
\item {\bfseries\sffamily TODO} Johnson, Jessica Marie. \emph{Wicked Flesh}
\label{sec:org567f0b0}
Johnson takes something very constraining--marriage and baptism
records--and turns it into a mechanism for reading intimacy. 

Johnson examines existing structures for modes of resistance that
emerge from the logics of those structures. Johnson takes something
very constraining, that is, the social structures available to African
and African-descent women who lived during and sometimes operated
within the slave-trading 18th century. She studies civil records of
marriage and baptism for the ways they enable new productions of
intimacy, which is a practice of freedom.

Johnson’s project addresses the problem of data, which is the problem
of the archive in the Atlantic World. Her methodology is to close-read
the civil records to create a “promiscuous accounting of blackness not
as bondage and subjection, but as future possibility” (10). The
records, on their own, tell a story of bondage and subjection to
power. Johnson’s job is to construct a narrative around these records,
about the “ways black women sought out profane, pleasurable, and
erotic entanglements as practices of freedom” (12). In “Markup
Bodies,”another piece by this author, Johnson argues that data on its
own re-enacts the commodification of black bodies---that black bodies
are reduced to data points. In order to subvert the reduction of
bodies to “data,” data has to be supplemented with “black digital
practice,” which are the rich histories of black embodiment, the
narrative that surrounds the data, which accompanies the database.

How are the records themselves constituted and structured? Databases
themselves are far from neutral (as Johnson undoubtedly would
agree). They are constructed, structured data. I’m curious about the
formal aspects of this database, in the decisions made not only in
their creation but in their preservation. What if we think of these
records as their own data format, and examine the structures /
boundaries of that format?

Is there something within this structure that allows us to read an
emergent freedom?
\item Johnson, Jessie Marie. "Markup Bodies"
\label{sec:org7a3e345}
Questions "data" as neutral and stable, in slavery's archive. Rather,
data replicates death and commodification. Proposes a "black digital
practice" (Johnson "Markup Bodies" 58) to challenge the reproduction
of black death and commodofication.

Digital tools "mark up the bodies and requantify the lives of people
of african descent" (Johnson "Markup Bodies" 59). 

Digitizing is replicating. Commodifying. 

The Atlantic Slave Trade Database example: the database alone is
replicating traumas: "metrics in minutae niether lanced historical
trauma nor bridged the gap between the past itself and the search for
redress" (Johnson "Markup Bodies" 62).

"transmutation of black flesh into integers and fractions" (Johnson
"Markup Bodies" 65). 

This data is "corrupted" and needs to be reworked. 

Black digital practice as "screams in the archive" (Johnson "Markup
Bodies" 71). To feel pain, center the unquantifiable. 

\item {\bfseries\sffamily TODO} Nakamura et al. \emph{Race After the Internet}
\label{sec:orga6e3ff3}
\item Roderick A. Ferguson, "Queer of Color Critique"
\label{sec:orgf5366ed}
\url{https://oxfordre.com/literature/view/10.1093/acrefore/9780190201098.001.0001/acrefore-9780190201098-e-33}

Queer of Color Critique is reactung against the collapsing of
difference in identity politics--queer theory was creating identities
for things in a way that incorporates/ignores intra-group
differences. Intersectionality disrupts the pluralism of
multiculturalism. 

"To the extent that queer studies understood sexuality to be singularly
constituted, the field betrayed its own investments in Eurocentric
presumptions of uniformity"

Building of Norma Alarcon, Kimberle Williams Crenshaw. Showing how
QOCC appears in Munoz.

"Queer of color critique began as a U.S.-based critical discourse
responding to the circumstances of migration, neoliberal state and
economic formations, and the developments of racial knowledges and
subjectivities about sexual and gender minorities within the United
States. As a convergence with and a departure from queer studies,
queer of color critique signaled the ways in which the dominant
literary, philosophical, and aesthetic engagements with queer
sexuality distanced themselves from the study of race and from
politico-economic concerns."

"As an effort designed to address connections between race, sexuality,
and political economy, queer of color critique had to begin by
confronting a founding limitation of queer studies, a limitation that
obscured the very connections that queer of color critique was
interested in exposing. That limitation had to do with an initial
ambivalence within queer studies about the connections that sexuality
has to other modes of difference\ldots{}.This presumed equivalence and
authenticity imply a liberal pluralism aimed at including racial and
gender subjects into existing normative institutions and systems."

"Consider, for instance, Norma Alarcón’s critique of the identity
politics of Anglo-American feminism in her 1990 essay “The Theoretical
Subjects of This Bridge Called My Back.” In that essay Alarcón
considers the epistemological impact of women of color feminism, in
general, and the 1981 anthology This Bridge Called My Back: Writings
by Radical Women of Color. For Alarcón, the overall work of women of
color feminists and the volume are significant not because they
offered a “rainbow theory” or a discourse of authenticity. They were
significant because they called into question the implicit subject of
Anglo-American feminism. As Alarcón put it, “[The] most popular
subject of Anglo-American feminism is an autonomous, self-making,
self-determining subject who first proceeds according to the logic of
identification with regard to the subject of consciousness, a notion
usually viewed as the purview of man, but now claimed for women.”"

"For Alarcón, This Bridge and women of color feminism analyzed the
ways in which patriarchal and liberal notions of agency account for
the ideological structure of Anglo-American feminism. In doing so,
Alarcón produces a devastating critique of how Anglo-American feminism
attempted to base feminist authenticity on a replication rather than a
repudiation of heteropatriarchal agency."

"Kimberlé Williams Crenshaw’s classic article “Mapping the Margins:
Intersectionality, Identity Politics, and Violence against Women of
Color” provided another example of how intersectional work was a
critique of authenticity politics. For instance, in her discussion of
identity politics, she argued, “The problem with identity politics is
not that it fails to transcend differences, as some critics charge,
but rather the opposite—that it frequently conflates or ignores
intra-group differences.”11 Crenshaw offered intersectionality as a
critique of the ways in which race and gender were politicized as
consistent with ideologies of discreteness and authenticity. As she
stated, “Feminist efforts to politicize experiences of women and
antiracist efforts to politicize people of color have frequently
proceeded as though the issues and experiences they each detail occur
in mutually exclusive terrains.”12 In designating race and gender as
categories that are constituted in relation to each other as well as
other social differences, Crenshaw presented intersectionality as a
means for disrupting discourses of authenticity and providing
alternatives to those discourses."

"Instead of arguing for the authenticity of categories of race and
gender, women of color feminism and its intersectional strategies can
be productively interpreted as attempts to show the limits of
authenticity discourses and the ways that those discourses
marginalized women of color. Contrary to the notion that modes of
difference can be understood singularly, women of color feminists
advanced arguments that posited those modes as necessarily constituted
in relation to one another, for historical and political reasons."

"Muñoz defines the hermeneutical properties of disidentification by
arguing, “For the critic, disidentification is the hermeneutical
performance of decoding mass, high, or any other cultural field from
the perspective of a minority subject who is disempowered in such a
representational hierarchy.”

\item {\bfseries\sffamily TODO} Roderick A. Ferguson’s Aberration in Black: Toward a Queer of Critique
\label{sec:org0bff1e6}
"The book attempted to demonstrate the ways in which the discipline’s
liberal regard for African Americans was precisely the mechanism for
constructing African American culture as outside of the normative
boundaries of the archetypal Western subject and the citizen-subject
of the United States."

\item {\bfseries\sffamily TODO} Kyla Wazana Tompkins, "Intersections of Race, Gender, and Sexuality: Queer of Color Critique"Kyla Wazana Tompkins, "Intersections of Race, Gender, and Sexuality: Queer of Color Critique"
\label{sec:org98d5ca8}
\url{https://d1wqtxts1xzle7.cloudfront.net/73757732/0b6ddb31eb5e591a549808f0f452a8d68c09-with-cover-page-v2.pdf?Expires=1646781190\&Signature=OuVOtUFXdNpSWcIb0OzaWCTvKU60kmsBPodlB-76fzil6fIDRhROktvC6ItdoYsmz\~UPnML8hqkysQapuoUahCh50umsSd\~bfYSn4dLRqO-zUZFQTR7LMNE-yia\~GE0o-6vQZWRSZRkrUWaONc-rdfud9xSL3FATOXDDt9QxIhneVPEIS\~m5wR5Cf7KLY7uPGok9Qm9XuEcUA85amkfgufkVNvVgKwxyLnGa92URE3pMW29v53gN96SGEKoB8s-PFXi44AS12Vr39hB3rZy5bD25I\~D0yzZWMczDwJRbv02u\~fGNX15H4ZHOPYZy-OVeqSadyy1cRfH8SIhOdulRrw\_\_\&Key-Pair-Id=APKAJLOHF5GGSLRBV4ZA}

\item {\bfseries\sffamily TODO} Blocket, Reginald. "Thinking with Queer of Color Critique: A
\label{sec:org9879d7b}
Multidimensional Approach to Analyzing and Interpreting Data

\item {\bfseries\sffamily TODO} Kim Gallon, "Making a Case for the Black Digital Humaniites"
\label{sec:org0272606}
\url{https://dhdebates.gc.cuny.edu/read/untitled/section/fa10e2e1-0c3d-4519-a958-d823aac989eb\#\#ch04}
\end{enumerate}

\subsubsection{Contemporary Textual Scholarship}
\label{sec:orgbbecf5d}
\begin{enumerate}
\item {\bfseries\sffamily DONE} Earhart, Amy E. "Models of Digital Documentation: The 19th-Century Concord Digital Archive," Documentary Editing: Journal of the Association for Documentary Editing, Volume 31: 2010 ISSN 0196-7134
\label{sec:org2ef754e}
Drawing from Textual Scholarship (Jerome McGann and Ken Price) to
apply to the way we think about race.  Following McGann in "Imagining
what we don't know", she says "you must build the archive to learn
what you need to know" (Earhart 40). What she has is not an archive, a
repository, but a "search-and-display space for materials located on
disparate servers" (40-41).

\item {\bfseries\sffamily DONE} Earhart, Amy E. "Can Information Be Unfettered? Race and the New
\label{sec:orgf7443ac}
Digital Humanities Canon" \emph{Debates in the Digital Humanities 2016}.
\url{https://dhdebates.gc.cuny.edu/read/untitled-88c11800-9446-469b-a3be-3fdb36bfbd1e/section/cf0af04d-73e3-4738-98d9-74c1ae3534e5} 

Increased access to technology has not increased representation of
marginalized subjects or texts. How to create inclusive structures for
digitization projects. Earhart emphasizes accomodation of
technological standards (like the TEI) and institutional affiliation
(like DH centers and funding) to projects that are DIY. She also
emphasizes "theoretical work to do in teh selection, editing, and
technological manipulation of our materials" (316).

What we need is messiness:
\begin{itemize}
\item Cites Martha Nell Smith about DH practitioners fleeing cultural
\end{itemize}
criticism, sheltering in the objectivity of the computer.
\begin{itemize}
\item Earhart says that we have a structural problem, more than a problem
\end{itemize}
of selection, in DH (314). 

Begins by discussing the digital canon. Early excitement
(technodeterminism) about access to material has not borne out the
dissolution of hegemonic high (white) culture.

DH as a field is slow to incorporate cultural criticism, ala Alan
Liu (310). 

See: 
\begin{itemize}
\item Charles Chestnutt Archive
\item Schomburg Library for 19thC Black Women Writers
\item 19th Century Digital Concord Archive (Earhart's project)
\end{itemize}

IN order for projects to actually succeed, they need affiliation and
funding, as well as rigorous technical standards. For example, \emph{NINES}
with the TEI. (315).

With her own project, the struggles with encoding multiple racial
identities over a single person: "how to appropriately apply
technological standards to shifting constructions of race" (316). 

\item {\bfseries\sffamily TODO} Julia Flanders. “From Modeling to Interpretation.” Bits That
\label{sec:org6905c94}
Matter, \url{https://juliaflanders.wordpress.com/}. Accessed 18 Aug. 2021.
\item {\bfseries\sffamily TODO} Michelle Schwartz and Constance Crompton. “Remaking History:
\label{sec:org6c96be1}
Lesbian Feminist Historical Methods in the Digital Humanities.” Bodies
of Information: Intersectional Feminism and Digital Humanities, edited
by Elizabeth Losh and Jacqueline Wernimont, University of Minnesota
Press, 2018,
\url{https://dhdebates.gc.cuny.edu/read/untitled-4e08b137-aec5-49a4-83c0-38258425f145/section/5c06c277-b9c1-4caf-a81c-a6c201e08a5a}.

\item {\bfseries\sffamily TODO} Gailey, Amanda. 2011. “Rethinking Digital
\label{sec:org2e7d8a6}
Editing Practices to Better Address Non-Canonical Texts," \emph{Documentary
Editing: Journal of the Association for Documentary Editing}, Volume
32: 2011 ISSN
0196-7134. \url{https://digitalcommons.unl.edu/cgi/viewcontent.cgi?article=1011\&context=docedit}  

\item {\bfseries\sffamily TODO} Smith, Martha Nell. “The Human Touch Software of the Highest
\label{sec:org49d0105}
Order: Revisiting Editing as Interpretation.” Textual Cultures,
vol. 2, no. 1, 2007, pp. 1–15. JSTOR,
\url{http://www.jstor.org/stable/30227853}.

In 2005, at the STS meeting, "that a rigid set of orthodoxies, a
"right" way of doing edito rial business, need not inform our
practices in order for them to be princi- pled, rigorous, and reliably
according to standard" (1). 

\item Piez, Wendell. "TEI in LMNL: Implications for Modeling." /Journal of
\label{sec:org1f5cd31}
the Text Encoding Initiative/. Isseue 8, December 2014 - December 2015
Selected Papers from the 2013 TEI
Conference. \url{https://doi.org/10.4000/jtei.1337} 
\item {\bfseries\sffamily TODO} La Fontaine, Robin. “Representing Overlapping Hierarchy as
\label{sec:orgc395294}
Change in XML.” Presented at Balisage: The Markup Conference 2016,
Washington, DC, August 2 - 5, 2016. In Proceedings of Balisage: The
Markup Conference 2016. Balisage Series on Markup Technologies,
vol. 17 (2016). \url{https://balisage.net/Proceedings/vol17/html/LaFontaine01/BalisageVol17-LaFontaine01.html} 
\item Non-XML approaches for non-hierchical markup
\label{sec:org3a1d6ce}
\url{https://tei-c.org/release/doc/tei-p5-doc/en/html/NH.html\#NHNX} 

\item {\bfseries\sffamily TODO} TEI 2019 conference papers
\label{sec:org7be5c75}
\url{https://gams.uni-graz.at/context:tei2019} 
\item {\bfseries\sffamily TODO} TEI 2022 conference program
\label{sec:org7c51b5e}
\url{https://www.conftool.pro/tei2022/sessions.php} 
\item {\bfseries\sffamily TODO} \emph{Scholarly Editing} journal
\label{sec:orge006414}
\url{https://scholarlyediting.org/}
\end{enumerate}
\subsubsection{TEI projects}
\label{sec:orgde17993}
\begin{enumerate}
\item {\bfseries\sffamily TODO} Editing the Eartha M White Collection
\label{sec:orgff37a1a}
\url{https://unfdhi.org/earthawhite/about-the-project} 
Editorial statement:
\url{https://unfdhi.org/ewproject/content/ew\_index.xml\#doubtsandconcerns}
\item The Life Histories Collection
\label{sec:org6af680e}
\url{https://www.pitjournal.unc.edu/comment/6} 

\item {\bfseries\sffamily DONE} Design, Development, and Documentation: Hacking TEI for Black
\label{sec:org514e0ae}
Digital Humanities, Jessica H. Lu and Caitlin Pollock.
\url{https://mith.umd.edu/digital-dialogues/dd-fall-2019-jessica-h-lu-caitlin-pollock/}

"Black DH Schema" project. TEI as a tool that can be critiqued and
disrupted, that centers black people, lives, cultures, rather than
relegates to the margins. Can there be a schema that supports and
encourages black dh work, confirm and amplify black humanity rather
than perpetuate power structures. 

"Praxis over product", citing Catherine Knight Steele. 
3 directions:
\begin{itemize}
\item Using black history, culture and life to critique TEI standards.
\item Examining how TEI reinscribes reductive categories of humanity,
marking black people as objects.
\item How TEI can open an opportunity for recognition and support for
Black invention, creativity, play.
\end{itemize}

Creating a statement of values about rejecting white supremacists
hierarchies that normalize white modes of writing/scholarship.

creating elements and attributes that:
\begin{itemize}
\item cite black women
\item chosen names vs imposed names, unnamed.
\item land acknowledgement - place name
\item tags for code-switching
\item tags for places beyond geographical point, "the corner," "kitchen
table", digital spaces.
\item tags that reflect complexity of black kinship and family, including
role names.
\end{itemize}

\item {\bfseries\sffamily DONE} Elisa Beshero-Bondar, "Black DH and a Challenge in Document Data Modeling
\label{sec:org8068d0f}
Anna Julia Cooper's Responses to the Survey of Negro College
Graduates"
\url{https://slides.com/elisabeshero-bondar/ajcsurvey-keydh} 
Also, on sex and gender in orlando:
\url{https://slides.com/elisabeshero-bondar/gendertei\#/3/3/0} 

\item {\bfseries\sffamily TODO} \href{https://chesnuttarchive.org/}{Charles Chesnutt Archive}
\label{sec:org863f498}

From "Encoding Guidelines": "the current TEI guidelines do not offer
standards for the encoding of printed page proofs with handwritten
corrections" 

Current TEI guidelines for project:
\url{https://docs.google.com/spreadsheets/d/e/2PACX-1vQfOJpBPdLRedtoRCahMXHVn3TSZgMALfIJhQLS-DjC25qjAz0L4x10ngNB5Ts7YLlj7XkkB5V\_A5o4/pubhtml?gid=0\&single=true} 

Specific guidelines for galley proofs: 
\url{https://chesnuttarchive.org/pdf/cwca\_galley\_proof\_encoding\_guidelines.pdf} 

\item \href{https://libguides.nypl.org/african-american-women-writers-of-the-19th-Century}{African American Women Writers of the 19th Century}
\label{sec:org6642b4f}
\end{enumerate}


\section{outline}
\label{sec:org13a4ac7}
\subsection{intro / question}
\label{sec:org1b67be4}
How do editorial practices with TEI engage queerness? How might they
approach queerness and schematization?
\begin{itemize}
\item The answer has to do with examining TEI structure for the way it
perpetuates certain assumptions.
\item But not only to examine this structure, but also to offer methods
for subverting it.
\end{itemize}

Tinseley's call to the material. 
\subsection{textual scholarship \& queer historiography}
\label{sec:orgaa6238b}
\begin{itemize}
\item summary of my work on \emph{Dorian Gray}
\item summary of productive vs restorative approaches. The question of
historical queerness mirrors that of editorial intent
\end{itemize}

\subsection{the TEI schema}
\label{sec:org959584f}
\begin{itemize}
\item But my perspective has been insufficient, going from boundedness to
heirarhcy.
\item TEI is about adding descriptive information to text. It's about a
kind of qualitative markup. This should give the standard some room
to accomodate complexity. "conceptual", "declarative", "logical",
"structural", and "semantic" markup.
\item But the governing structure is a nesting structure. That means that
elements are necessarily contained within a hierarchy. One root
element encapsulates them all, and variation is only possible by
means of extension from the root.
\end{itemize}

\subsubsection{Dominance is totalizing, or bureaucratic}
\label{sec:org6355656}
Hierarchical conflicts in text documents often have to do with a
contradiction between the physical structure of a document (page or
line of text, for example) and the semantic structure (sentence,
chapter, for example. These issues often occur in poetry, where the
conflict (and it is a lovely conflict) between metrical and sytactic
structures, such as the use of enjambment to continue a sentence
beyond the end of a line. 

XML researcher and creator of LMNL Jeni Tennison explores how markup
might prioritize containment as well as suggest dominance
relationships between elements. She distinguishes between containment
and dominance:
\begin{quote}
When you’re talking about overlapping structures, it’s useful to make
the distinction between structures that \emph{contain} each other and
structures that \emph{dominate} each other. Containment is a happenstance
relationship between ranges while dominance is one that has a
meaningful semantic. A page may happen to contain a stanza, but a poem
domainates the stanzas that it contains. Tennison, "Overlap,
Containment, and Dominance"; emphasis original
\end{quote}
One is more power-neutral, the other implies power
relationships. Tennison, who "want[s] to see if we can get away with
not having hierarchy as a fundamental part of the information model,"
hinks about XML schemas based on containment rather than dominance,
where the tags mark ranges describing start and stop points rather
than elements which are nested (Tennison, "Essential Hierarhcy"). The
problem is that dominance is at time necessary. For example,
"Analysing the way in which the syntactic (sentence/phrase) structure
overlaps with the prosodic (stanza/line) structure is one important
way in which you can analyse a poem" (Tennison, "Overlap, Containment,
and Dominance"). She points out that there is no easy solution for
including dominance into data structures. You either have to place
\emph{everything} within a tree structure, use milestones (like TEI does,
severing elements, effectively), or develop a special syntax to
indicate dominance (which models like GODDAG and LMNL. Dominance
structures, in other words, are either totalizing, or they are
excessively bureaucratic.

TEI guidelines, module 16 on "Linking, Segmentation, and Alignment,"
describes various methods for encoding information that is not
hierarchic or linear, which are based on the W3C XPointer framework
and generally use the \texttt{@xml:id} attribute, pointers, blocks, segments,
anchors, correspondence, alignment, synchronization, aggregation,
alternation, sequesteration, marginalization, among others. Module 20,
"Non-hierarchical Structures" 

\subsubsection{Form cannot be separated from politics}
\label{sec:orgd075ac4}

\subsection{QOCC intervention in queer editorial work}
\label{sec:org5e29c59}
\subsubsection{Qualitative expansion:}
\label{sec:orgee8d03a}
Jessica Marie Johnston's way of thinking about data replicating past
traumas/oppressions; in order to get more from the data, we need
qualitative expansion, "black digital practice". A narrativizing over
the constraints of the archival record.

\subsubsection{Creating inclusive structures}
\label{sec:org881ef37}
\begin{itemize}
\item Problem with the canon, it's still too white. Projects need
affiliation and funding in order to succeed. How can we replicate
this for DYI projects? (Earhart)
\item creating elements and attributes that reflect black life (Jessica
H. Lu and Caitlin Pollock).
\end{itemize}

\subsection{conclusion: the future of editing}
\label{sec:org46e5880}

\section{draft}
\label{sec:org435a041}
\subsection{intro}
\label{sec:orgc5484a2}
This paper considers the potential alignment between a rigidly
structured and constraining editorial format, the TEI, and a
strategically nebulous amalgam of identities and politics expressed by
the designation of queerness. It considers how textual editing
practices with the TEI might reflect or engage with queerness as
theorized by Queer Studies. It then proposes a possible future for
developing editorial methodologies with TEI to mark up non-normative
identities and sexualities in text.

This project begins with a self-reflection on my training in Textual
Scholarship, and past work that interweaves critical conversations in
Textual Scholarship with those from the field of Queer
Historiography. I then point out how my involvement in the work of
marking up queerness work led me to overlook the imbrication of
queerness with race. In my attention to queer possibilites for
resistance, I failed to notice the ways in which institutional
whiteness operates as an unmarked but structuring force in both
textual editing practices and Queer Studies. To correct that
oversight, this paper examines the rise of Queer of Color's \emph{Critique}
on Queer Studies, and how the work on the archive of slavery offers
models for recognizing the structuring modes of knowledge innate to
recovery and preservation practices, not to mention data formats. I
then turn to the TEI schema, to interrogate how its hierarchical and
bounded structure might be problematized or re-worked to engage some
of the more invisible forces that determine meaning-making. I close by
offering some examples of this work in contemporary editorial
projects, and some reflections on future possibilties. 

What did I find? That TEI projects need:
\begin{itemize}
\item Digital workflows for collaboration.
\item using nonexperts (students) bc of lack of funding
\end{itemize}

\subsection{textual scholarship and queer historiography}
\label{sec:orgf6b1260}
I begin with my own trajectory of thinking on the subject. From my
first days in graduate school, intimidated by the heady atmosphere of
class discussion that swelled with theories I had never heard of in
the language of convoluted abstraction, I found refuge in a course
about the field and practice of Textual Scholarship. The focus on the
text as a material object, as something that takes up space in the
world and that I could literally touch, formed what seemed to me to be
the foundation of all textual criticism. Textual editing methodologies
like the TEI grounded this intellectual work in the physical, minute
labor of transcription and markup. It was at that time that I was
introduced to Jerome McGann's \emph{Radiant Textuality: Litery Studies
After The Worldwide Web} (2001). McGann's position on the role of
digital tools in literary scholarship, that they ought to work as
"prosthetic extension[s] of that demand for critical reflection," as
opposed to, for example, a means of preserving or establishing some
truth about a text, solidifed something essentially human about the
critial process: that it is fundamentally creative (McGann 2001, 18).

With this certainty in mind, I pursued editorial projects on queer
texts, with a focus in genetic editing methods. In one project, I
turned to Oscar Wilde's manuscript of \emph{The Picture of Dorian Gray}
(1890), a holograph draft that he revised heavily before sending it
for publication in \emph{Lippincott's Monthly Magazine} on June 20,
1890.\footnote{Calado, F. d., (2022) "Encoding Queer Erasure in Oscar Wilde’s
\emph{The Picture of Dorian Gray}." See Wilde and Frankel, pp. 40--54, for
a more complete accounting of the preparation of the typescript for
publication.} Of Wilde's revisions, I focused on those concerning the
homoerotic innuendos between the story's three main characters, Basil
Hallward, Lord Henry Wotten, and the eponymous Dorian Gray. After
transcribing the manuscript, I used TEI with the purpose of surfacing
homoerotic elements that Wilde had excised or obscured during his
revision process, and marked these revisions according to four main
themes, which I called: "intimacy," "beauty," "passion," and
"fatality," with the additional values of "inconclusive" and
"illegible" for moments when numerous pen strokes hindered efforts at
transcription. I decided on these themes because they expressed
general patterns for the revisions, including the stifling of
emotional tension, physical affection, references to beauty and
passion, and to the obsessive and self-destructive effects of
infatuation. In addition to marking up conceptual changes to the
manuscript, I also noted the physical changes, that is, the presence
and number of Wilde's pen strokes as he eliminated spans of text.  

I set out in this project to explore the limitations of the TEI format
by attempting to mark up information that I knew would provoke the
bounds of the tags themselves. The queer themes of this text cannot be
separated into discrete elements, but rather flows in a spectrum of
smooth information (Flanders). I also sought to engage the difficulty
of the conceptual information with another register entirely, the
physical register of Wilde's pen strokes across the pages.

My editorial work on this project unearthed, as I had expected it to,
a resistence to the demand for fixity in the TEI schema. The
boundedness of the TEI format, which encapsultates data within a
structured set of tags, struggled against the porous perimeters of
these queer themes in the text. The encoding of pen strokes adds
another layer of difficulty, for the number of strokes over a single
revision often failed to map with the themes. While some of the
editorial were straightforward, for example, sections where Wilde's
pen slashes through evidence of physical contact can be marked as
"intimacy," like when Basil "tak[es] hold of [Lord Henry's] hand"
(Wilde 9), or when Dorian's "cheek just brushed [Basil's] cheek"
(Wilde 20), others were more difficult. For example, dampening the
intimacy sometimes had the attendant effect of mitigating the sense of
fatality that surrounds Basil's attraction to Dorian. This occurs in
one striking moment from the dialogue, as Basil struggles to impart to
Lord Henry the effect of his passion for Dorian Gray. The original
line in the manuscript reads: "Lord Henry hesitated for a moment. ‘And
what is that?' he asked, in a low voice. ‘I will tell you,' said
Hallward, and a look of pain came over his face. ‘Don't if you would
rather not,' murmured his companion, looking at him" (9). In the
revised version, Lord Henry "laugh[s]" rather than "hesistate[s]," he
no longer speaks "in a low voice," and his "look of pain" is
neutralized into "an expression of perplexity." These changes, which
work to lighten a particularly tense display of intimacy, also have
the effect of obscuring Basil's internal suffering to evoke the theme
of "fatality." The additional challenge of marking up the strokes in
this section also reinforces the limitations of TEI's nested
structure: while the word "look" is struck so heavily that the number
of strokes is inconclusive, the word "pain" contains a single
stroke. With the TEI, it is impossible to mark the number of strokes
for each word without separating the single revision into two
instances.

For this project, I drew together my editorial principles from across
the disparate fields of Textual Scholarship and Queer Historiography,
who within their own spheres of influence are having what I perceive
to be an analogous debate about the role of recovery in dealing with
documents from the past. Until the popularization of the digital
editing methods in the 90s and early 2000s, Textual Scholarship tended
to privilege the editor as a recoverer or preserver of text. Prominent
editors like Ronald B. McKerrow promoted authorial intention as the
highest criterion for editorial decisions. His position was
subsequently developed through the work of Walter W. Greg, who
expanded the critic's purview beyond the single copy-text, and then to
Fredson Bowers and Thomas Tanselle who proposed an eclectic editing
practice that could distill authorial intention from multiple
sources.\footnote{Should I include McKerrow, Bowers, and Tanselle in the References?} Toward the end of the 20th century, the emphasis on
authorial intention, what I call the "restorative approach," begins to
shift in the wake of new tools that could multiply, rather than
narrow, the potential forms that editorial work might take. Here, the
work of Donald F. McKenzie challenged the idea a single text could
ever represent an "ideal" version. Rather, he explains, the text is a
product of a network of agencies, what he calls a "sociology" of
texts. Electronic environments opened a space for representing textual
variation unhindered by the limitations of the codex
format. Electronic editor Jerome McGann's work on the \emph{Rossetti
Archive} showcased how digital design might display textual variation
to suggest a flexible and reflexive approach to editing. Opposed to
the restorative approach of their predecessors, I call McKenzie and
McGann's approach "productive," for the way it subscribes the text to
new configurations that opens up questions of formal significance. 

My reading of Queer Historiography, as a field, finds analogous
debates between restorative and productive tendencies. Here, the
question of recovering authorial intention relates to queer
historiographical debates defining queerness as an identity across
time. One side of the debate, the "unhistoricists," argue that
queerness in the past cannot be scrutinized in the present without
subscribing it to a teleology that effectively normalizes its
essential alterity, its quality of resistence that constitutes
queerness. The historicists, by contrast, maintain that queerness can
be traced as a historically situated phenomenon, and requires
historical specificity in order to be legible. Valerie Traub explains
that "Queer's free-floating, endlessly mobile, and infinitely
subversive capacities may be strengths--allowing queer to accomplish
strategic maneuvers that no other concept does--but its principled
imprecision implies analytic limitations" (Traub, 2013: 33). In other
words, the term "queer", if applied ahistorically, would lose its
descriptive value. Offering a solution to this problem, Heather Love
proposes a critical methodology that, I argue, evokes the "productive
approach." Her study of negative affects associated with "social
refusal and the denigration of homosexual love," such as shame, anger,
hatred, or disappointment, refrains from attempting to "fix" or
"resolve" them into contemporary perspectives of identity and
desire. Instead of attempting to incorporate queerness into
contemporary perspectives, this method, called "feeling backward,"
attends to the ways that queerness eludes containment or
resolution. For editing the unweildy revision history of Wilde's text,
Love's method offered a means for pursuing editing as a formal
experiment by which "success would constitute failure" (Love, 2009:
51). In other words, the project of editing becomes a reflexive,
formal experiment that aligns with "productive" approaches to challege
through exposure the boundedness of the TEI structure.

\subsection{The TEI schema: formal experiment is the problem.}
\label{sec:orgf3e2e56}
This formal experiment, however "productive" in its refusal against
the search for origins,\footnote{Jacques Derrida, "Archive Fever."} now seems insufficient. The problem with
TEI as a data model, I'm coming to realize, goes beyond the
boundedness of its elements and into the hierarchical document model,
which perpetuates implicit power relations within the document. As a
tree structure, the parent elements will always dominate the
subordinate ones, with the root element at the top of the
hierarchy. The document model creates a power structure that not only
encapsulates but dominates information.

TEI tags and attributes add \emph{qualitative} information to data. Markup
can support a number of encoding approaches, including "conceptual,"
"declarative," "logical," "structural," and "semantic." Although it
has a default set of tags and attributes, TEI has been praised for its
customizability, which it inherits from its parent language, XML
(eXtensible Markup Language). Individual encoding projects often
create their own custom schemas that reflect the priority of each
document. And, due to the collective nature of the TEI, which is
developed and maintained by the TEI Consortium, the guidelines are
continually updated to accomodate the needs of encoders. 

At the most recent annual TEI Conference and Members Meeting in 2022,
Elisa Beshero-Bondar, Helena Bermúdez Sabel, Raffaele Viglianti, and
Janelle Jenstad presented their work on developing a \texttt{<gender>}
element for the TEI guidelines. Their proposal for a new \texttt{<gender>}
element, which is under review for the next release of the TEI
Guidelines, emphasizes both the expressive and theoretical potential
and the possible risks of reifying normative cultural biases for
representing gender. However, as other projects seeking to encode
plural or multiple gender ontologies have explained, the main issue is
not the presence or absence of a gender marker itself, but the larger
hierarchical structure that contains the marker.\footnote{See Thain, "Perspective: Digitizing the Diary--Experiments in
Queer Encoding" and Caughie et al, "Storm Clouds on the Horizon:
Feminist Ontologies and the Problem of Gender".} Queer gender
ontology can take many forms, some of which can be delineated and
contained within a capacious enough set of tags, such as distinct
\texttt{<gender>} and \texttt{<sex>} tags, as proposed by Beshero-Bondar and her
team. Other formulations, however, are too porous or blurry to be
separated into distinct categories. In the latter case, the problem
goes deeper than the tag's boundedness, to challenge the hierarchical
structure of the TEI document model. For example, Beshero-Bondar and
her colleagues explain that in their work in revising the existing
\texttt{<sex>} element,
\begin{quote}
Unexpectedly, we found ourselves confronting the Guidelines’
prioritization of personhood in discussion of sex, likely stemming
from the conflation of sex and gender in the current version of the
Guidelines. In revising the technical specifications describing sex,
we introduced the term “organism” to broaden the application of sex
encoding. We leave it to our community to investigate the fluid
concepts of gender and sex in their textual manifestations of
personhood and biological life. Beshero-Bondar et al.
\end{quote}
While the new element, \texttt{<gender>}, gives the team some capacity to
represent gender as distinct from sex, the tags nonetheless exist
within a hierarchical structure that perpetuates a particular dynamic
of domination which asserts that "sex" must be subservient to some
concept of personhood. The proposed solutions to this problem, which
include revising \texttt{<person>} to \texttt{<organism>} and even \texttt{<entity>},\footnote{martindholmes, "New \texttt{<entity>} and \texttt{<listEntity>} elements are
needed \#2341."  Github, \url{https://github.com/TEIC/TEI/issues/2341}.}
maintain these systems of dominance.

\subsection{{\bfseries\sffamily TODO} Dominance is totalizing, or bureaucratic}
\label{sec:org3e2bab2}
The problem with XML is that the document model is totalizing. 


\section{Works}
\label{sec:org6e60698}
E. Beshero-Bondar, et al. “Revising Sex and Gender in the TEI Guidelines.” TEI Conference and Members’ Meeting 2022, 2022, \url{https://www.conftool.pro/tei2022/sessions.php}.

Calado, F. d., (2022) “Encoding Queer Erasure in Oscar Wilde’s "The
Picture of Dorian Gray"”, Open Library of Humanities 8(1). doi:
\url{https://doi.org/10.16995/olh.6407q}

Caughie, P L, Datskou, E and Parker, R 2018 ‘Storm Clouds on
the Horizon: Feminist Ontologies and the Problem of Gender.' Feminist
Modernist Studies 1.3, 230--242. DOI: 10.1080/24692921.2018.1505819

Gaboury, J 2018 ‘Becoming NULL: Queer relations in the
excluded middle.' Women \& Performance: a Journal of Feminist Theory
28.2, 143--158. DOI: 10.1080/0740770X.2018.1473986

Goldberg, J and Menon, M 2005 ‘Queering History.' PMLA, 120.5,
1608--1617. DOI: 10.1632/003081205X73443

martindholmes, "New \texttt{<entity>} and \texttt{<listEntity>} elements are
needed \#2341."  Github, \url{https://github.com/TEIC/TEI/issues/2341}.

McGann, Jerome J. 2001. Radiant Textuality: Literature After the World
Wide Web. New York: Palgrave. Print.

Tennison, Jeni. "Overlap, Containment, and Dominance." \emph{Jeni's
Musings}. Mon,
2008-12-06. \url{http://www.jenitennison.com/2008/12/06/overlap-containment-and-dominance.html}  

Tennison, Jeni. "Essential Hierarhcy." \emph{Jeni's
Musings}. 2008-12-09
\url{https://web.archive.org/web/20111230054946/http://www.jenitennison.com/blog/node/96}

Thain, M 2016 ‘Perspective: Digitizing the Diary --
Experiments in Queer Encoding,' Journal of Victorian Culture, 21.2,
226--241. DOI: 10.1080/13555502.2016.1156014

Traub, V 2013 ‘The New Unhistoricism in Queer Studies.' PMLA,
128.1, 21--39. DOI: 10.1632/pmla.2013.128.1.21

Sperberg-McQueen, C. M. and Claus Huitfeldt. “GODDAG: A Data Structure
for Overlapping Hierarchies.” DDEP/PODDP (2000).

Wilde, O 1889--90 MA 883. The Picture of Dorian Gray:
Original Manuscript. Morgan Library \& Museum, New York, NY.

Wilde, O and Frankel, N 2011 The Picture of Dorian Gray: An
Annotated, Uncensored Edition. Cambridge: Harvard University Press. DOI:
10.4159/harvard.9780674068049
\end{document}
